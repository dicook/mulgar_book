
\index{missing values}

\index{shadow matrix|see{missing values}}

\chapter{Missing Values}

% One newpage, to preserve a

Values are often missing in data, for several reasons.  Measuring
instruments fail, samples are lost or corrupted, patients do not show
up to scheduled appointments, and measurements may be deliberately
censored if they are known to be untrustworthy above or below certain
thresholds.  When this happens, it is always necessary to evaluate the
nature and the distribution of the gaps, to see whether a remedy must be
applied before further analysis of the data.  If too many values are
missing, or if gaps on one variable occur in association with other
variables, ignoring them may invalidate the results of any analysis
that is performed.  This sort of association is not at all uncommon
and may be directly related to the test conditions of the study.  For
example, when measuring instruments fail, they often do so under
conditions of stress, such as high temperature or humidity.  As
another example, a lot of the missing values in smoking cessation
studies occur for those people who begin smoking again %\cite{BDSL06}
and silently withdraw from the study, perhaps out of discouragement or
embarrassment.

\index{missing values!imputation}

%Imputing missing values
In order to prepare the data for further analysis, one remedy that is
often applied is imputation, that is, filling in the gaps in the data
with suitable replacements.  There are numerous methods for imputing
missing values. Simple schemes include assigning a fixed value such as
the variable mean or median, selecting an existing value at random, or
averaging neighboring values. More complex distributional approaches
to imputation start with the assumption that the data arises from a
standard distribution such as a multivariate normal, which can then be
sampled to generate replacement values. See \citeasnoun{Sh97} for a
description of multiple imputation and \citeasnoun{LR87}
for a description of imputation using multivariate distributions.

% Di:  add reference to MANET, and GGobi papers on missing values.
% Di: look at Hmisc (Harrell) package also for imputing missings!!

In this chapter, we will discuss the power of visual methods at both
of these stages: diagnosing the nature and seriousness of the
distribution of the missing values (Sect.~\ref{miss-explore}) and
assessing the results of imputation (Sect.~\ref{miss-impute}).  The
approach follows those described in \citeasnoun{SB98} and
\citeasnoun{UHS96}.

% This paragraph is out of place now but includes important points.  
% Some of it belongs in the imputation section.  Does some of it 
% belong in the recap?

%Unfortunately, most books on missing values do not emphasize the use
%of visual methods to assess the results of imputation.  Perhaps this
%oversight is partly due to the fact that traditional graphical
%methods, either static or interactive, have not offered many
%extensions designed for working with missing values.  It is common to
%exclude missings from plots altogether; this is unsatisfactory, as
%we can guess from the example above. The user usually needs to assign
%values for missings to have them incorporated in plots. Once some
%value has been assigned, the user may lose track of where the missingsn
%once were.  More recent software offers greater support for the
%analysis of missings, both for exploring their distribution and for
%experimenting with and assessing the results of imputation methods
%(see \citeasnoun{SB98,UHS96}).

\section{Background}

%What is missingness?

% Examples
% Classification of Missing Stucture according to Rubin & Little
% MCAR, MAR, MNAR
% examples for different types
% concept of ignorable/non-ignorable missings

\index{missing values!missing completely at random (MCAR)}
\index{missing values!missing at random (MAR)}
\index{missing values!missing not at random (MNAR)}
\index{missing values!ignorable}

% classification of missingness
Missing values are classified by \citeasnoun{LR87} into three
categories according to their dependence structure: missing completely
at random (MCAR), missing at random (MAR), and missing not at random
(MNAR).  Only if values are MCAR is it considered safe to ignore
missing values and draw conclusions from the set of
complete cases, that is, the cases for which no values are missing.
Such missing values may also be called \Term{ignorable}.

The classification of missing values as MCAR means that the
probability that a value is missing does not depend on any other
observed or unobserved value; that is, $P(missing | observed,
unobserved) = P(missing)$.  This situation is ideal, where the chance
of a value being missing depends on nothing else. It is impossible to
verify this classification in practice because its definition includes
statements about unobserved data; still, we assume MCAR if there is
dependence between missing values and observed data.
%Since it
%depends on knowledge about unobserved data, however, it is difficult
%to verify in practice.  
%The practical application of this
%classification is that, if there is no dependence between missing
%values and the observed data, then we will assume MCAR.

In classifying missing values as MAR, we make the more realistic
assumption that the probability that a value is missing depends only
on the observed variables; that is, $P(missing | observed, unobserved)
= P(missing | observed)$. This can be verified with data.  For values
that are MAR, some structure is allowed as long as all of the
structure can be explained by the observed data; in other words, the
probability that a value is missing can be defined by conditioning on
observed values. This structure of missingness is also called
\Term{ignorable}, since conclusions based on likelihood methods are
not affected by MAR data.

\index{missing values!non-ignorable}

Finally, when missing values are classified as MNAR, we face a
difficult analysis, because $P(missing | observed, unobserved)$ cannot
be simplified and cannot be quantified.  \Term{Non-ignorable} missing
values fall in this category. Here, we have to assume that, even if we
condition on all available observed information, the reason for
missing values depends on some unseen or unobserved information.

Even when missing values are considered ignorable we may wish to
replace them with imputed values, because ignoring them may lead to a
non-ignorable loss of data.  Consider the following constructed data,
where missings are represented by the string NA, meaning ``Not
Available.''  There are only 5 missing values out of the 50 numbers in
the data:

\begin{center}
\begin{tabular}{r@{\hspace{.5em}}|@{\hspace{.3em}}cccccc}
Case \T \B & $X_1$ & $X_2$ & $X_3$ & $X_4$ & $X_5$ \\
\hline
\T1& NA & 20 & 1.8 & 6.4 & $-$0.8 \\
2 & 0.3 & NA & 1.6 & 5.3 & $-$0.5 \\
3 & 0.2 & 23 & 1.4 & 6.0 & NA \\
4 & 0.5 & 21 & 1.5 & NA & $-$0.3 \\
5 & 0.1 & 21 & NA & 6.4 & $-$0.5 \\
6 & 0.4 & 22 & 1.6 & 5.6 & $-$0.8 \\
7 & 0.3 & 19 & 1.3 & 5.9 & $-$0.4 \\
8 & 0.5 & 20 & 1.5 & 6.1 & $-$0.3 \\
9 & 0.3 & 22 & 1.6 & 6.3 & $-$0.5 \\
\B10& 0.4 & 21 & 1.4 & 5.9 & $-$0.2 \\
\end{tabular}
\end{center}

\noindent Even though only 10\% of the numbers in the table are
missing, 100\% of the variables have missing values, and so do 50\% of
the cases.  A complete case analysis would use only half the data.

Graphical methods can help to determine the appropriate classification
of the missing structure as MCAR, MAR or MNAR, and this is described in
Sect.~\ref{miss-explore}.  Section~\ref{miss-impute} describes how
the classification can be re-assessed when imputed values are checked
for consistency with the data distribution.

\section{Exploring missingness}\label{miss-explore}

One of our first tasks is to explore the distribution of the missing
values, seeking to understand the nature of ``missingness'' in the
data.  Do the missing values appear to occur randomly, or do we detect
a relationship between the missing values on one variable and the
recorded values for some other variables in the data?  If the
distribution of missings is not random, this will weaken our ability
to infer structure among the variables of interest.  It will be shown
later in the chapter that visualization is helpful in searching for
the answer to this question.

\subsection{Shadow matrix}~\label{miss-shadow}

\index{missing values!shadow matrix}

As a first step in exploring the distribution of the missing values,
consider the following matrix, which corresponds to the data matrix
above and has the same dimensions:

\begin{center}
\begin{tabular}{r@{\hspace{.5em}}|@{\hspace{.3em}}cccccc}
Case \T \B & $X_1$ & $X_2$ & $X_3$ & $X_4$ & $X_5$ \\\hline
\T1 & 1 & 0 & 0 & 0 & 0 \\
2 & 0 & 1 & 0 & 0 & 0 \\
3 & 0 & 0 & 0 & 0 & 1 \\
4 & 0 & 0 & 0 & 1 & 0 \\
5 & 0 & 0 & 1 & 0 & 0 \\
6 & 0 & 0 & 0 & 0 & 0 \\
7 & 0 & 0 & 0 & 0 & 0 \\
8 & 0 & 0 & 0 & 0 & 0 \\
9 & 0 & 0 & 0 & 0 & 0 \\
\B10 & 0 & 0 & 0 & 0 & 0 \\
\end{tabular}
\end{center}

In this binary matrix, 1 represents a missing value and 0 a recorded
value.  It is easier to see the positions of the missing values in
this simple version and to consider their distribution apart from the
data values.  We like to call this the ``shadow matrix.''

Sometimes there are multiple categories of missingness, in which case
this matrix would not simply be binary.  For example, suppose we were
conducting a longitudinal study in which we asked the same questions
of the same subjects over several years.  In that case, the missing
values matrix might include three values: 0 could indicate an answered
question, 1 that a survey respondent failed to answer a question, and
2 that a respondent died before the study was completed.

\index{datasets!\Data{Tropical Atmosphere-Ocean Array (TAO)}}

% Description of data example
As an example for working with missing values, we use a small subset
of the \Data{TAO} data: all cases recorded for five locations
(latitude 0\degree with longitudes 110{\degree}W and 95{\degree}W,
2{\degree}S with 110{\degree}W and 95{\degree}W, and 5{\degree}S with
95{\degree}W) and two time periods (November to January 1997, an El
Ni\~no event, and for comparison, the period from November to January
1993, when conditions were considered normal).  Load the data into R
and GGobi to investigate missingness:

\begin{verbatim}
> library(norm)
> library(rggobi)
> d.tao <- read.csv("tao.csv", row.names=1)
> d.tao.93 <- as.matrix(subset(
  d.tao,year==1993,select=sea.surface.temp:vwind))
> d.tao.97 <- as.matrix(subset(
  d.tao,year==1997,select=sea.surface.temp:vwind))
> d.tao.nm.93 <- prelim.norm(d.tao.93)
> d.tao.nm.93$nmis
sea.surface.temp         air.temp         humidity 
               3                4               93 
           uwind            vwind 
               0                0 
> d.tao.nm.97 <- prelim.norm(d.tao.97)
> d.tao.nm.97$nmis
sea.surface.temp         air.temp         humidity 
               0               77                0 
           uwind            vwind 
               0                0 
\end{verbatim}

There are 736 data points, and we find missing values on
three of the five variables (Table~\ref{TAO-varmiss}). 

%Missingness by variable
\begin{table}[htp]
\centering
\caption[Missing values on each variable]{Missing values on each variable}
% Add space between columns
\begin{tabular}{l@{\hspace{.3in}}rr}\hline

\multicolumn{1}{l}{Variable} \T \B & \multicolumn{2}{c}{Number of} \\
                          & \multicolumn{2}{c}{missing values} \\
  & \T 1993 & 1997 \\ 
\hline
\Vbl{sea surface temp} \T & 3 & 0\\
\Vbl{air temp} & 4 & 77\\
\Vbl{humidity} & 93 & 0\\
\Vbl{uwind} & 0 & 0\\
\Vbl{vwind} \B & 0 & 0\\\hline
\end{tabular}
\label{TAO-varmiss}
\end{table}

\noindent We are also interested in tabulating missings by case:

\begin{verbatim}
> d.tao.nm.93$r
    [,1] [,2] [,3] [,4] [,5]
274    1    1    1    1    1
  1    0    0    1    1    1
 90    1    1    0    1    1
  1    1    0    0    1    1
  2    0    0    0    1    1
> d.tao.nm.97$r
    [,1] [,2] [,3] [,4] [,5]
291    1    1    1    1    1
 77    1    0    1    1    1
\end{verbatim}

\noindent From Table \ref{TAO-casemiss}, we can see that most 
cases have no missing values (74.5\% in 1993, 79.1\% in 1997), and
less than a quarter of cases have one missing value (24.5\% in 1993,
20.9\% in 1997). In 1993 two cases are missing two values and two
cases have missing values on three of the five variables, \Vbl{sea
surface temp}, \Vbl{air temp} and \Vbl{humidity}.

%Missingness by case
\begin{table}[htp]
\centering
\caption[Distribution of the number of missing values on a
case]{Distribution of the number of missing values on a case.}
\begin{tabular}{r@{\hspace{1.2in}}r@{\hspace{.3in}}rp{0.2in}r@{\hspace{.3in}}r}
\hline

\multicolumn{1}{l}{No. of missings} \T &
  \multicolumn{2}{c}{1993} & & \multicolumn{2}{c}{1997} \\
\multicolumn{1}{l}{on a case} \B & 
  \multicolumn{1}{l}{No. of cases} & \multicolumn{1}{r}{\%} & &
  \multicolumn{1}{l}{No. of cases} & \multicolumn{1}{r}{\%} \\

\hline 
\T 3 & 2 & 0.5 & & 0 & 0\\ 
   2 & 2 & 0.5 & & 0 & 0\\ 
   1 & 90 & 24.5  & & 77 & 20.9\\ 
\B 0 & 274 & 74.5 & & 291 & 79.1\\
\hline
\end{tabular}
\label{TAO-casemiss}
\end{table}

To study the missingness graphically, load the data and set up the
colors and glyphs to reflect the \Vbl{year}, which corresponds to two
different climate conditions.

\begin{verbatim}
> gd <- ggobi(d.tao)[1]
> glyph_color(gd) <- ifelse(gd[,1]==1993,5,1)
> glyph_type(gd) <- ifelse(gd[,1]==1993,6,4)
\end{verbatim}

\subsection{Getting started: missings in the ``margins''}

The simplest approach to drawing scatterplots of variables with
missing values is to assign to the missings some fixed value outside
the range of the data, and then to draw them as ordinary data points
at this unusual location.  It is a bit like drawing them in the
margins, which is an approach favored in other visualization
software. In Fig.~\ref{missing}, the three variables with missing
values are shown.  The missings have been replaced with a value 10\%
lower than the minimum data value for each variable.  In each plot,
missing values in the horizontal or vertical variable are represented
as points lying along a vertical or horizontal line, respectively.  A
point that is missing on both variables appears as a point in the
lower left corner; if multiple points are missing on both, this point
is simply over-plotted.

%Comment for authors: This figure is a pair of plots: Humidity vs
%Air Temp, Sea Surface Temp vs Air Temp. The points are colored by
%year, as they are by default in the data.]

% Figure 1
\begin{figure}[htbp]
\centerline{\includegraphics[width=2.2in]{chap-miss/splots2.pdf}
            \includegraphics[width=2.2in]{chap-miss/splots1.pdf}}
\caption[Assigning constants to missing values]{Assigning constants
to missing values.  In this pair of scatterplots, we have assigned to
each missing value a fixed value 10\% below the variable minimum, so
the ``missings'' fall along vertical and horizontal lines to the left
and below the point scatter.  The green solid circles (the cluster
that has lower values of \Vbl{air temp}) represent data
recorded in 1993; the purple open circles show the 1997 data.}
\label{missing}
\end{figure}

What can be seen?  Consider the plot of \Vbl{air temp} vs.
\Vbl{sea surface temp}.  Not surprisingly, the two temperature
values are highly correlated as indicated by the strong linear pattern
in the plot; we will make use of this fact a bit later. We can also
see that the missings in that plot fall along a horizontal line,
telling us that more cases are missing for \Vbl{air temp} than
for \Vbl{sea surface temp}.  Some cases are missing for both,
and those lie on the point in the lower left corner.  The live plot
can be queried to find out how many points are over-plotted there. To
alleviate the problem of over-plotting, we have also jittered the values
slightly; i.e., we have added a small random number to the missing
values.  In this plot, we also learn that there are no cases missing
for \Vbl{sea surface temp} but recorded for \Vbl{air
temp} ~---~ if that were true, we would see some points plotted
along a vertical line at roughly \Vbl{sea surface temp} $ =
20$.  The right-hand plot, \Vbl{air temp} vs. \Vbl{humidity}, is
different: There are many cases missing on each variable but not
missing on the other.

Both pairwise plots contain the same two clusters of data, one for
1993 records (green filled circles) and the other for 1997, an El
Ni\~no year (purple open circles).  There is a relationship between
the variables and the distribution of the missing values, as we can
tell simply from the color of the missings.  For example, all
cases for which \Vbl{humidity} was missing are green, so we know they
were all recorded in 1993.  The position of the missings on
\Vbl{humidity} tells the same story, because none of them lie within
the range of \Vbl{air temp} in 1997.  We know already that, if we
excluded these cases from our analysis, the results would be
distorted: We would exclude 93 out of 368 measurements for 1993, but
none for 1997, and the distribution of \Vbl{humidity} is quite
different in those two years.

The other time period, in 1997, is not immune from missing values
either, because all missings for \Vbl{air temp} are in purple.

In summary, from these plots we have learned that there is dependence
between the missing values and the observed data values.  We will see
more dependencies between missings on one variable and recorded values
on others as we continue to study the data. At best, the missing values
here may be MAR (missing at random).

\subsection{A limitation}

Populating missing values with constants is a useful way to begin, as
we have just shown.  We can explore the data we have and begin
our exploration of the missing values as well, because these simple
plots allow us to continue using the entire suite of interactive
techniques.  Multivariate plots, though, such as the tour and parallel
coordinate plots are not amenable to this method.

Using fixed values in a tour causes the missing data to be mapped onto
artificial planes in $p$-space, which obscure each other and the main
point cloud. Figure~\ref{missing2} shows two tour views of \Vbl{sea
surface temp}, \Vbl{air temp}, and \Vbl{humidity} with
missings set to 10\% below minimum. The missing values appear as
clusters in the data space, which might be thought of as lying along
three walls of a room with the complete data as a scattercloud within
the room.

% show those walls?
% would it be possible to re-impute values for missings on the fly?
% - i.e. depending on s projection, set the missing values to 10% below the minimum.  
% That would keep them in their corner.

% Need several projections/plots to demonstrate the problem
 
% Figure 2
\begin{figure}[htb]
\centerline{
  \includegraphics[width=2.1in]{chap-miss/tour1.pdf}
  \includegraphics[width=2.1in]{chap-miss/tour2.pdf}
}
\caption[Tour views with missings set to 10\% below minimum]{Tour
views of \Vbl{sea surface temp}, \Vbl{air temp}, and
\Vbl{humidity} with missings set to 10\% below minimum. There appear
to be four clusters, but two of them are simply the cases that have
missings on at least one of the three variables.}
\label{missing2}
\end{figure}

% Figure 3
\begin{figure}[htb]
\centerline{\includegraphics[width=4.4in]{chap-miss/parcoords.pdf}}
\caption[Parallel coordinates plot with missings set to 10\% below
minimum]{Parallel coordinates of the five variables \Vbl{sea surface
temp}, \Vbl{air temp}, \Vbl{humidity}, \Vbl{uwind}, and
\Vbl{vwind} with missings set to 10\% below minimum. There are two
groups visible for \Vbl{humidity} in 1993 (green, the color drawn
last), but that is because a large number of missing values are
plotted along the zero line; for the same reason, there appear to be
two groups for \Vbl{air temp} in 1997 (purple).  }
\label{missing3}
\end{figure}

Figure~\ref{missing3} shows the parallel coordinate plot of \Vbl{sea
surface temp}, \Vbl{air temp}, \Vbl{humidity},
\Vbl{uwind}, and \Vbl{vwind} with missings set to 10\% below minimum.
If we did not know that the points along the zero line were the
missings, we could be led to false interpretations of the plot.
Consider the values of \Vbl{humidity} in 1993 (the green points, the
color drawn last), where the large number of points drawn at the zero
line look like a second cluster in the data.

% needs some conclusion along the line: plots with missing values 
% have to be interpreted with special care

When looking at plots of data with missing values, it is important to
know whether the missing values have been plotted, and if they have,
how they are being encoded.

\subsection{Tracking missings using the shadow matrix}

In order to explore the data and their missing values together, we
will treat the shadow matrix (Sect.~\ref{miss-shadow}) as data and
display projections of each matrix in linked windows, side by side.
In one window, we show the data with missing values replaced by
imputed values; in the missing values window, we show the binary
indicators of missingness.

Although it may be more natural to display binary data in area plots,
we find that scatterplots are often adequate, and we will use them
here.  We need to spread the points to avoid multiple over-plotting,
so we jitter the zeros and ones.  The result is a view such as the
left-hand plot in Fig.~\ref{miss-data}.  The data fall into four
squarish clusters, indicating presence and missingness of values
for the two selected variables.  For instance, the top right cluster
consists of the cases for which both variables have missing values,
and the lower right cluster shows the cases for which the horizontal
variable value is missing but the vertical variable value is present.

\index{brushing!linked}
Figure~\ref{miss-data} illustrates the use of the \Data{TAO} dataset
to explore the distribution of missing values for one variable with
respect to other variables in the data.  We have brushed in orange
squares only the cases in the top left cluster, where \Vbl{air
temp} is missing but \Vbl{humidity} is present.  We see in the
right-hand plot that none of these missings occur for the lowest
values of \Vbl{uwind}, so we have discovered another dependence
between the distribution of missingness on one variable and the
distribution of another variable.

We did not really need the missings plot to arrive at this observation;
we could have found it just as well by continuing to assign
constants to the missing values.  In the next section, we will continue
to use the missings plot as we begin using imputation.

\index{missing values!imputation}
\section{Imputation}\label{miss-impute}

Although we are not finished with our exploratory analysis of this subset
of the \Data{TAO} data, we have already learned that we need to
investigate imputation methods.  We have already learned that we will not
be satisfied with complete case analysis.  We cannot safely throw out
all cases with a missing value, because the distribution of the missing
values on at least two variables (\Vbl{humidity} and \Vbl{air temp})
is strongly correlated with at least one other data variable
(\Vbl{year}).

Because of this correlation, we need to investigate imputation
methods.  As we replace the missings with imputed values, though, we
do not want to lose track of their locations.  We want to use
visualization to help us assess imputation methods as we try them,
making sure that the imputed values have nearly the same distribution
as the rest of the data.

% Figure 4
\begin{figure}[htbp]
\centerline{\includegraphics[width=2.2in]{chap-miss/impute1a.pdf}
            \includegraphics[width=2.2in]{chap-miss/impute1b.pdf}
}
\caption[Exploring missingness in the \Data{TAO} data]{Exploring
missingness in the \Data{TAO} data.  The ``missings'' plot {\bf
(left)} for \Vbl{air temp} vs. \Vbl{humidity} is a jittered
scatterplot of zeroes and ones, where one indicates a missing value.
The points that are missing only on \Vbl{air temp} have been
brushed in orange.  In a scatterplot of \Vbl{vwind} vs. \Vbl{uwind}
{\bf (right)}, those same missings are highlighted.  There are no
missings for the very lowest values of \Vbl{uwind}.}
\label{miss-data}
\end{figure}

% Di: Use the same two plots in all the examples here, Air temp vs sst, and 
% air temp vs humidity.

\subsection{Mean values}

The most rudimentary imputation method is to use the variable mean to
fill in the missing values.  In the middle row of plots in
Fig.~\ref{impute5}, we have substituted the mean values for missing
values on \Vbl{sea surface temp}, \Vbl{air temp}, and
\Vbl{humidity}.  Even without highlighting the imputed values, some
vertical and horizontal lines are visible in the scatterplots.  This
result is common for any imputation scheme that relies on
constants. Another consequence is that the variance--covariance of the
data will be reduced, especially if there are a lot of missing values.

% Figure 5
\begin{figure}[htbp]
\centerline{\includegraphics[width=2.2in]{chap-miss/splots2.pdf}
            \includegraphics[width=2.2in]{chap-miss/splots1.pdf}}
\centerline{
  \includegraphics[width=2.2in]{chap-miss/impute4a.pdf}
  \includegraphics[width=2.2in]{chap-miss/impute4b.pdf}
}
\centerline{
  \includegraphics[width=2.2in]{chap-miss/impute2a.pdf}
  \includegraphics[width=2.2in]{chap-miss/impute2b.pdf}
}
\caption[Comparison of simple, widely used imputation
schemes]{Comparison of simple, widely used imputation schemes.
Missings in the margin, as in Fig.~\ref{missing} {\bf (Top row)}.
Missing values have been replaced with variable means, conditional on
\Vbl{year}, producing vertical and horizontal stripes in each cluster
{\bf (Middle row)}.  Missing values have been filled in by randomly
selecting from the recorded values, conditional on \Vbl{year} {\bf
(Bottom row)}. The imputed values are a little more varied than the
recorded data.  }
\label{impute5}
\end{figure}

\subsection{Random values}

It is clear that a random imputation method is needed to better
distribute the replacements.  The simplest method is to fill in the
missing values with some value selected randomly from among the
recorded values for that variable.  In the bottom row of plots in
Fig.~\ref{impute5}, we have substituted random values for missing
values on \Vbl{sea surface temp}, \Vbl{air temp}, and
\Vbl{humidity}. The match with the data is much better than when mean
values were used: It is difficult to distinguish imputed values from
recorded data! However, taking random values ignores any association
between variables, which results in more variation in values than occurs
with the recorded data. If you have a keen eye, you can see that
in these plots.  It is especially visible in the plot of \Vbl{sea
surface temp} and \Vbl{air temp}, for the 1997 values
(in purple): They are more spread, less correlated than the complete
data.

\index{brushing!linked}
The imputed values can be identified using linked brushing between the
missings plot and the plot of \Vbl{sea surface temp} vs.
\Vbl{air temp} (Fig.~\ref{impute4}). Here the values missing on 
\Vbl{air temp} have been brushed (orange rectangles) in the 
missings plot (left), and we can see the larger spread of the imputed
values in the plot at right. 

% Figure 6
\begin{figure}[htbp]
\centerline{
  {\includegraphics[width=2.0in]{chap-miss/impute3a.pdf}}
  {\includegraphics[width=2.0in]{chap-miss/impute3b.pdf}} }
\caption[Conditional random imputation]{Conditional random
imputation.  Missing values on all variables have been filled in using
random imputation, conditioning on drawing symbol.  The imputed values
for \Vbl{air temp} show less correlation with \Vbl{sea surface temp}
than do the recorded values.}
\label{impute4}
\end{figure}

\subsection{Multiple imputation}

A more sophisticated approach to imputation is to sample from a
statistical distribution, which may better reflect the variability in
the observed data. Common approaches use regression models or
simulation from a multivariate distribution.

To use regression, a linear model is constructed for each of the
variables containing missings, with that variable as the response, and
the complete data variables as explanatory variables. The distribution
of the residuals is used to simulate an error component, which is
added to the predicted value for each missing, yielding an imputed
value. Many models may be required to impute all missing
values. For example, in the \Data{TAO} data, we would need to fit a
model for \Vbl{sea surface temp}, \Vbl{air temp}, and
\Vbl{humidity} separately for each \Vbl{year}.  This can be laborious!

Simulating from a multivariate distribution yields imputed values from
a single model. For the \Data{TAO} data, it might be appropriate to
simulate from a multivariate normal distribution, separately for each
\Vbl{year}. 

\index{missing values!multiple imputation}

With either approach, it is widely acknowledged that one set of
imputed values is not enough to measure the variability of the
imputation.  Multiple imputed values are typically generated for each
missing value with these simulation methods, and this process is
called \Term{multiple imputation}.


\index{R package!\RPackage{norm}}
\index{R package!\RPackage{Hmisc}}

R packages, such as \RPackage{norm} by \citeasnoun{NS06} or
\RPackage{Hmisc} by \citeasnoun{Ha06}, contain multiple imputation
routines.  To view the results in GGobi, we can dynamically load
imputed values into a running GGobi process.  This next example
demonstrates how to impute from a multivariate normal model using R
and how to study the results with GGobi. Values are imputed separately
for each year.

To apply the light and dark shades used in Fig.~\ref{multimp} to the
GGobi process launched earlier, select \Button{Color Schemes} from
GGobi's \Button{Tools} menu and the \Button{Paired 10} qualitative
color scheme before executing the following lines:

%# missings are a darker shade of the same color
\begin{verbatim}
> gcolor <- ifelse(gd[,1]==1993,3,9)
> glyph_color(gd) <- gcolor
> ismis <- apply(gd[,4:8], 1, function(x) any(is.na(x)))
> gcolor[ismis] <- gcolor[ismis]+1
> glyph_color(gd) <- gcolor
\end{verbatim}

\noindent Make a scatterplot of \Vbl{sea surface temp} and 
\Vbl{air temp}. At this stage, the missing values are still 
shown on the line below the minimum of the observed points.  In the
next step, the missing values are imputed multiply from separate
multivariate normal distributions for each of the two years.

\begin{verbatim}
> rngseed(1234567) 
> theta.93 <- em.norm(d.tao.nm.93, showits=TRUE)
Iterations of EM: 
1...2...3...4...5...6...7...8...9...10...11...12...13...14...
15...16...17...18...19...20...21...22...23...
> theta.97 <- em.norm(d.tao.nm.97, showits=TRUE)
Iterations of EM: 
1...2...3...4...5...6...7...8...9...10...11...12...13...14...
> d.tao.impute.93 <- imp.norm(d.tao.nm.93, theta.93,
    d.tao.93)
> d.tao.impute.97 <- imp.norm(d.tao.nm.97, theta.97,
    d.tao.97)
\end{verbatim}
\newpage
\begin{verbatim}
> gd[,"sea.surface.temp"] <- c(
    d.tao.impute.97[,"sea.surface.temp"],
    d.tao.impute.93[,"sea.surface.temp"])
> gd[,"air.temp"] = c(
    d.tao.impute.97[,"air.temp"],
    d.tao.impute.93[,"air.temp"])
> gd[,"humidity"] = c(
    d.tao.impute.97[,"humidity"],
    d.tao.impute.93[,"humidity"])
\end{verbatim}

% Figure 7
\begin{figure}[htbp]
\centerline{
  {\includegraphics[width=2.2in]{chap-miss/multimpute1.pdf}}
  {\includegraphics[width=2.2in]{chap-miss/multimpute2.pdf}}
}
\centerline{
  {\includegraphics[width=2.2in]{chap-miss/multimpute3.pdf}}
  {\includegraphics[width=2.2in]{chap-miss/multimpute4.pdf}}
}
\caption[Two different imputations using simulation from a
multivariate normal distribution of all missing values]{Two different
imputations using simulation from a multivariate normal distribution
of all missing values.  In the scatterplot of \Vbl{air temp} vs.
\Vbl{sea surface temp} the imputed values may have different means
than the complete cases: higher \Vbl{sea surface temp} and lower
\Vbl{air temp}.  The imputed values of \Vbl{humidity} look quite
reasonable.  }
\label{multimp}
\end{figure}

\noindent The missings are now imputed, and the scatterplot of \Vbl{sea
surface temp} and \Vbl{air temp} should look like the
one in Fig.~\ref{multimp}. The imputation might make it necessary to
re-scale the plot if values have fallen outside the view; if so, 
use the \Button{Rescale} button in the \Button{Missing Values} panel.

% Figure 8
\begin{figure}[htbp]
\centerline{{\includegraphics[width=2in]{chap-miss/multimpute5.pdf}}
  {\includegraphics[width=2in]{chap-miss/multimpute6.pdf}}}
\caption[Tour projection of the data after multiple imputation]{Tour
projection of the data after multiple imputation of \Vbl{sea surface
temp}, \Vbl{air temp}, and \Vbl{humidity}.}
\label{multimp-tour}
\end{figure}

\index{animation}
\noindent Figures~\ref{multimp} and \ref{multimp-tour} show plots of 
the data containing
imputed values resulting from two separate simulations from a
multivariate normal mixture.  In this coloring, green and purple still
mean 1993 and 1997, but now light shades represent recorded values and
dark shades highlight the missing values ~---~ now imputed.

The imputed values look reasonably good. There are some small
differences from the recorded data distribution: Some imputed values
for \Vbl{sea surface temp} and \Vbl{air temp} in 1997
are higher than the observed values, and some imputed values for
\Vbl{humidity} in 1993 are higher than the observed values.


% Something about MNAR structure here
% And something about checking for normal samples, refer to inference section

\section{Recap}

In this chapter, we showed how to use graphical methods to develop
a good description of missingness in multivariate data. Using the
\Data{TAO} data, we were able to impute reasonable replacements for
the missing values and to use graphics to evaluate them.

\index{brushing!linked} The data has two classes, corresponding to two
distinct climate patterns, an El Ni\~no event (1997) and a normal
season (1993).  We discovered the dependence on \Vbl{year} as soon we
started exploring the missingness, using missings plotted in the
margins.  Later we discovered other dependencies among the missings
and the wind variables using linked brushing between the missings plot
(shadow matrix) and other views of the data. These suggest the missing
values should be classified as MAR and therefore ignorable, which
means that imputation is likely to yield good results.

It was clear that we had to treat these two classes separately in
order to get good imputation results, and we imputed values using
multiple imputation, simulating from two multivariate normal
distributions.

After studying the imputed values, we saw that they were not perfect.
Some of the imputed values for \Vbl{air temp} and \Vbl{sea
surface temp} were higher than the observed values. This
suggests that the missingness is perhaps MNAR, rather than MAR. Still,
the imputed values are close to the recorded values.  For practical
purposes, it may be acceptable to use them for further analysis of the
data.

% Characterizing missing structure
% Imputing
% Strategy, splitting by year.
% Missings in the margins, to explore
% shadow matrix
% driving imputation from R

\section*{Exercises}
\addcontentsline{toc}{section}{Exercises}

\begin{enumerate}
%\item
%Describe the distribution of the two wind variables and the two
%temperature variables conditional on the distribution of missing
%values in \Vbl{humidity}.
% I think this is identical to the next question!
\item
Describe the distribution of the wind and temperature variables
conditional on the distribution of missing values in \Vbl{humidity},
using brushing and the tour.
\item
For the \Data{Primary Biliary Cirrhosis (PBC)} data:
\begin{enumerate}
\item Describe the univariate distributions of complete cases for 
\Vbl{chol}, \Vbl{copper}, \Vbl{trig}, and \Vbl{platelet}. 
What transformations might be used to make the distributions more
bell-shaped? Make these transformations, and use the transformed
data for the rest of this exercise.
% log, log, log, none
\item Examine a scatterplot matrix of \Vbl{chol}, \Vbl{copper}, \Vbl{trig}, 
\Vbl{platelet} with missing values plotted in the margins. 
  \begin{enumerate}
     \item Describe the pairwise relationships among the four variables.
     \item Describe the distribution between missings and non-missings
         for \Vbl{trig} and \Vbl{platelet}. % Looks fairly similar.
  \end{enumerate}
\item Generate the shadow matrix, and brush the missing values a
different color.
\item Substitute means for the missing values, and examine the result
in a tour.  What pattern is obvious among the imputed values? 
% Flat plane
\item Substitute random values for the missings, and examine the result
in a tour.  What pattern is obvious among the imputed values? 
% A few become outliers
\item In R, generate imputed values using multiple imputation. 
Examine different sets of imputed values in the scatterplot matrix. Do
these sets of values look consistent with the data distribution?
\item Using spine plots, examine each categorical variable
(\Vbl{status}, \Vbl{drug}, \Vbl{age}, \Vbl{sex}), checking for
associations between the variable and missingness.
% slightly more proportion of missings for females
\end{enumerate}
\end{enumerate}


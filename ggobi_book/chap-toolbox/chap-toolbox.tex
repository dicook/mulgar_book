%\setcounter{chapter}{3}

\chapter[The Toolbox]{The Toolbox}~\label{toolbox}

% Includes two newpage commands.  The first protects
% a section header from being orphaned, and the second fixes
% a bad word break at the end of a page.

The tools used to perform the analyses described in this book come
largely from two ``toolboxes.''  One is R, which we use extensively
for data management and manipulation, modeling, and static plots.
Since R is well documented elsewhere, both in books
\cite{Dalgaard2002,VR02,Murrell2005} and on the web ({\tt
R-project.org}), we will say very little about it here.

Instead, we emphasize the less familiar tools drawn from GGobi, our
other major toolbox: a set of direct manipulations that we apply to a
set of plot types. With these plot types and manipulations we can
construct and interact with multiple views, linked so that an action
in one can affect them all.  The examples described throughout the
book are based on these tools.

\section{Introduction}

\index{datasets!\Data{Australian Crabs}}

It will be helpful to have a shorthand for describing the information
that is used to generate a plot and what is shared between plots in
response to user manipulations.  We will introduce this notation using
the \Data{Australian Crabs} data, an excerpt of which is shown in
Table~\ref{crabs-data}.

%\bigskip
\begin{table}[h]
\centering
\caption{Example data table, an excerpt from \Data{Australian Crabs}}
\label{crabs-data}
\begin{center}

\begin{tabular}{r@{\hspace{.2in}}l@{\hspace{.2in}}l@{\hspace{.2in}}r@{\hspace{.2in}}r@{\hspace{.2in}}r@{\hspace{.2in}}r@{\hspace{.2in}}r}

\hline

\multicolumn{1}{l}{Crab} \T \B & \multicolumn{1}{l}{\Vbl{species}} &
\multicolumn{1}{l}{\Vbl{sex}} & \multicolumn{1}{c}{\Vbl{frontal}} &
\multicolumn{1}{c}{\Vbl{rear}} & \multicolumn{1}{c}{\Vbl{carapace}} &
\multicolumn{1}{c}{\Vbl{carapace}} & \multicolumn{1}{c}{\Vbl{body}} \\

 & & & 
\multicolumn{1}{c}{\Vbl{lobe}} & 
\multicolumn{1}{c}{\Vbl{width}} & 
\multicolumn{1}{c}{\Vbl{length}} & 
\multicolumn{1}{c}{\Vbl{width}} & 
\multicolumn{1}{c}{\Vbl{depth}} \\\hline

1\T& blue & male & 8.1 & 6.7 & 16.1 & 19.0 & 7.0 \\
2& blue & male & 8.8 & 7.7 & 18.1 & 20.8 & 7.4 \\
3& blue & male & 9.2 & 7.8 & 19.0 & 22.4 & 7.7 \\
4& blue & male & 9.6 & 7.9 & 20.1 & 23.1 & 8.2 \\
51& blue & female & 7.2 & 6.5 & 14.7 & 17.1 & 6.1 \\
52& blue & female & 9.0 & 8.5 & 19.3 & 22.7 & 7.7 \\
53& blue & female & 9.1 & 8.1 & 18.5 & 21.6 & 7.7 \\
101& orange & male & 9.1 & 6.9 & 16.7 & 18.6 & 7.4 \\
102& orange & male & 10.2 & 8.2 & 20.2 & 22.2 & 9.0 \\
151& orange & female & 10.7 & 9.7 & 21.4 & 24.0 & 9.8 \\
152& orange & female & 11.4 & 9.2 & 21.7 & 24.1 & 9.7 \\
153\B& orange & female & 12.5 & 10.0 & 24.1 & 27.0 & 10.9\\\hline
\end{tabular}
\end{center}
\end{table}
%\bigskip


The data table is composed of two categorical variables (\Vbl{species}
and \Vbl{sex}) and five real-valued variables (\Vbl{frontal lobe},
\Vbl{rear width}, \Vbl{carapace length}, \Vbl{carapace width}, and
\Vbl{body depth}). The real-valued variables are the physical
measurements taken on each crab, and the categorical variables are
class or grouping labels for each crab.  This distinction is important
for visualization as well as for analysis, and it is made throughout
the examples in the chapters to follow.  \index{plots!of real-valued
variables} \index{plots!of categorical variables} Graphics for
real-valued variables and categorical variables differ substantially,
and they complement one another.  For example, when both variable
types are present, it is common to plot a pair of real-valued
variables in a scatterplot and a single categorical variable in a
barchart, and to color the points in the scatterplot according to
their level in the categorical variable.

The 
% real-valued 
variables in a data table can be described algebraically using a
matrix having $n$ observations and $p$ variables denoted as:
%Why did we take real-valued out here?

\begin{eqnarray*}
\blX =
[\blX_1~\blX_2~\dots~\blX_p] = \left[ \begin{array}{cccc}
X_{11} & X_{12} & \dots & X_{1p} \\
X_{21} & X_{22} & \dots & X_{2p}\\
\vdots & \vdots &  & \vdots \\
X_{n1} & X_{n2} & \dots & X_{np} \end{array} \right]_{n\times p}
\end{eqnarray*}

\noindent For some analyses, we consider real-valued variables
separately.  As a subset of the \Data{Crabs} data, the values in the
table form a $12 \times 5$ matrix, $\blX_{12\times 5} = \left[ \blX_1,
\dots, \blX_5 \right]$.  We will use this shorthand notation when we
discuss methods that are applied only to real-valued variables. It is
particularly useful for describing tour \index{tour} methods,
which are presented in Sect.~\ref{tours}.

For the \Data{Crabs} data, we are interested in understanding the
variation in the five physical (real-valued) variables, and whether
that variation depends on the levels of the two categorical
variables. In statistical language, we may say that we are interested
in the \Term{joint distribution} of the five physical measurements
\Term{conditional} on the two categorical variables. A plot of one
column of numbers displays the \Term{marginal distribution} of one
variable. Similarly a plot of two columns of the data displays the
marginal distribution of two variables. Ultimately we want to describe
the distribution of observed values in the five-dimensional space of
the physical measurements.

Building insight about structure in high-dimensional spaces starts
simply.  We start with univariate and bivariate plots, looking for
low-dimensional structure, and work our way up to multivariate plots
to seek relationships among several variables.  Along the way, we use
different approaches to explore real-valued and categorical variables.

With larger datasets it is convenient to store the data table in a
database and to access it with database queries. For example, this SQL
command to subset the \Data{Crabs} data would return the \Vbl{frontal lobe}
values of the female Blue crabs:

\begin{verbatim}
SELECT frontal_lobe FROM Australian Crabs 
WHERE species=blue AND sex=female
\end{verbatim}

\noindent Database querying is a way to conceptually frame direct
manipulation methods [see, for example, \citeasnoun{AWS91} and
\citeasnoun{BMMS91}]. The above SQL query can be constructed
graphically using interactive brushing\index{brushing}, and the
response can be presented visually, as described in Sect.~\ref{brush}.

This chapter starts with plot types, and continues with a discussion
of plot manipulation.  The section on plot types is divided into
subsections corresponding to the number of data dimensions displayed,
and it ends with a brief discussion of the simultaneous use of
multiple views.  The section on direct manipulation describes brushing
and labeling (often using linked views), interactive view scaling, and
moving points.  We will close the chapter by mentioning some interesting
tools in other software packages.

\index{datasets!\Data{Tips}}

We will focus on two datasets to demonstrate the tools: \Data{Australian
Crabs} and \Data{Tips} (described in Chap.~\ref{intro}).  \Data{Tips}
will be used to illustrate categorical variable plots, and
\Data{Crabs} will be used to illustrate real-valued variable plots and
plots combining categorical and real-valued variables.  A few other
datasets will make brief appearances.

% Should we introduce categorical data summaries here  too? Since 
% we switch from Crabs to Tips data

\section{Plot types}
\index{plots!univariate}
\subsection{Univariate plots}

One-dimensional (1D) plots such as histograms, box plots, or dot plots are
used to examine the marginal distributions of single variables. What
shape do the values show ~---~ unimodal or multimodal, symmetric or
skewed?  Are there clumps or clusters of values? Are a few values
extremely different from the rest?  Observations like these about a
data distribution can only be made by plotting the data.

Each univariate plot type uses one column of the data matrix,
$\blX_i, i=1, \dots, p$. 

\subsubsection{Real-valued variables}

Two types of univariate plots for real-valued variables 
are regularly used in this book, each one a variant of the dot plot.
%: the textured dot plot and an average
%shifted histogram (ASH) dot plot. 
In a dot plot, one row of data generates one point on the plot,
explicitly preserving the identity of each observation.  This is
useful for linking information between plots using direct
manipulation, which is discussed later in the chapter.  We also use
histograms, where bar height represents either the count or the relative
frequency of values within a bin range; we usually use the count.
These are examples of ``area plots,'' in which observations are
aggregated into groups and each group is represented by a single
symbol (in this case, a rectangle).  Area plots are especially useful
when the number of points is large enough to slow down the computer's
response time or to cause over-plotting problems.

The plots in Fig.~\ref{dotfig} show three different univariate plots
for the column $\blX_3$ (\Vbl{frontal lobe}), for a subset of the data
corresponding to the female crabs of the Blue species.  The
first plot is a histogram, and the next two are dot plots.

% We could have these on one line, but we'd have to reduce
% the plot size considerably.  Maybe it will work better if
% Hadley can reduce the margin requirements for ggplot.

\begin{figure}[htp]
\centering
\centerline{
 \includegraphics[width=2.0in]{chap-toolbox/hist-FL-blue-female.pdf}
}
\centerline{
 \includegraphics[width=2.0in]{chap-toolbox/dot-FL-blue-female.pdf}
 \includegraphics[width=2.0in]{chap-toolbox/ASH-FL-blue-female.pdf}
}
\caption[Univariate displays]{Univariate displays: Histogram, textured
dot plot, and average shifted histogram of \Vbl{frontal lobe} of the
female crabs of the Blue \Vbl{species}.  We see bimodality in the
distribution of values, with many cases clustered near 15 and another
cluster of values near 12.}
\label{dotfig}
\end{figure}

\index{dot plot}
\index{jittering}
\index{dot plot!stacked}
\index{dot plot!jittered}
\index{dot plot!textured}
In a dot plot, each case is represented by a small circle (i.e., a dot).
Since there are usually several dots with the same or similar values,
some method must be used to separate those dots.  They may be
\Term{stacked}, which results in a plot that is much like a histogram, or
they may be \Term{jittered}, which adds a small amount of noise to each
point.  Jittering sometimes results in clumps of points that give the
false impression of meaningful clustering, so the \Term{textured dot
plot} was developed \cite{TukeyTukey90} to spread the data more
evenly.  When there are three or fewer cases with the same or similar
data values, those points are placed at constrained locations; when
there are more than three cases with the same value, the points are
positioned using a combination of constraint and randomness.

\index{average shifted histogram (ASH)}

The last plot in Fig.~\ref{dotfig} is an \Term{average shifted
histogram} or \Term{ASH} plot, devised by \citeasnoun{Scott92}.  In
this method, several histograms are calculated using the same bin
width but different origins, and the averaged bin counts at each data
point are plotted.  His algorithm has two key parameters: the number
of bins, which controls the bin width, and the number of histograms to
be computed.  The effect is a smoothed histogram, and since it can
shown as a dot plot, it is a histogram that can be linked case by case
to other scatterplots.

\subsubsection{Categorical variables}

\index{barchart}
\index{barchart!spine plot}

One-dimensional categorical data is commonly plotted in a bar chart,
where the height of the bar is proportional to the number of cases in
each category.  Figure~\ref{barchart} displays a barchart of \Vbl{day}
in the \Data{Tips} data: There were fewer diners on Friday than on the
other three days for which we have data. An alternative to a bar chart
is a spine plot, where the category count is represented by the width
of the bar rather than by the height.  When there is only one
variable, the spine plot tells the same story as does the barchart;
its special value emerges when plotting more than one categorical
variable, as should become clear shortly.

\begin{figure}[htp]
\centerline{
 {\includegraphics[width=2.3in]{chap-toolbox/cat9.pdf}}
 {\includegraphics[width=2.3in]{chap-toolbox/cat10.pdf}}}
\caption[Barchart and spine plot of the \Data{Tips} data] {Barchart
and spine plot of \Vbl{day} in the \Data{Tips} data.  There are
relatively fewer records on Friday, as we can see in both plots.
Friday's bar is the shortest one in the barchart {\bf (left)} and the
narrowest one in the spine plot.}
\label{barchart} 
\end{figure}

\subsection{Bivariate plots}

\index{plots!bivariate}
% Move out of the real-valued variables section -- expand a bit?

Plots of two variables are used for examining the joint distribution
of two variables. This pairwise relationship is often a marginal
distribution of multivariate data.

\subsubsection{Real-valued variables}

\index{scatterplot}

We use two-dimensional (2D) scatterplots to see the relationship
between two real-valued variables.  This relationship may be linear,
non-linear, or non-existent.  We also look for deviations from
dependence, such as outliers, clustering, or heterogeneous variation.

An example is shown in the two plots in Fig.~\ref{scat}.  In the first
plot, the two plotted variables are a subset of the five physical
measurements of the female Blue \Data{Australian Crabs}.  The plot
shows a strong linear relationship between \Vbl{rear width} and
\Vbl{frontal lobe}.  In the second plot, we have shown a
transformation of the data: the same two variables after they have
been sphered using a principal component transformation, which has
removed the linear relationship.
\index{principal component analysis}

Scatterplots are often enhanced by overlaying density information
using contours, color, or a grayscale. If one variable can be
considered a response and the other an explanatory variable, we may
add regression curves or smoothed lines.

\begin{figure}[htp]
\centerline{
 {\includegraphics[width=2in]{chap-toolbox/scat-FL-RW-blue-female.pdf}}
 {\includegraphics[width=2in]{chap-toolbox/scat-PC1-PC2-blue-female.pdf}}}
\caption[Bivariate scatterplots for the Blue female \Data{Australian
Crabs}]{Bivariate scatterplots for the Blue female \Data{Australian
Crabs}.  The scatterplot of \Vbl{rear width} against \Vbl{frontal
lobe} {\bf (left)} shows a strong linear relationship.  The plot of
principal component 2 against principal component 1 shows that the
principal component transformation removes the linear relationship.}
\label{scat}
\end{figure}

\subsubsection{Categorical variables}

\index{spine plot}

A spine plot can also be defined as a one-dimensional (1D) mosaic
plot.  When we add a second categorical variable, we can use the
resulting 2D mosaic plot to explore the dependence between the two
variables.  The left-hand plot in Fig.~\ref{mosaic} shows a mosaic
plot of \Vbl{day} and \Vbl{sex} from the \Data{Tips} data.  The bars
are split into gray and orange sections representing male and female
bill payers.

\index{mosaic plot}
\index{barchart!stacked}

The heights of the areas representing males increase from Thursday to
Sunday, which shows the change in the relative proportion of male to
female bill-payers.  If this data were displayed as a stacked bar
chart, as in the right-hand plot, it would be quite difficult to
compare the relative proportions of each bar; that is, it would be
hard to decide whether the proportion of female bill-payers varies
with day of the week.

A mosaic plot can handle several categorical variables; it shows the
frequencies in an $n$-way contingency table, each one corresponding to
a rectangle with proportional area.  It is constructed by repeatedly
subdividing rectangular areas.  In the example, the size of each
rectangular area in the example is proportional to the number of
cases having level $i$ of the \Vbl{day} and level $j$ of \Vbl{sex}.
If we were to add a third categorical variable, such as \Vbl{smoker},
we would divide each of the eight rectangles by a horizontal division.

\begin{figure}[htp]
\includegraphics[width=2.0in]{chap-toolbox/cat12.pdf}
\includegraphics[width=2.0in]{chap-toolbox/cat11.pdf}
\caption[A mosaic plot of \Vbl{day} and \Vbl{sex} for the \Data{Tips}
data]{A mosaic plot of \Vbl{day} and \Vbl{sex} for the \Data{Tips}
data, in which male and female bill payers are colored gray and
orange, respectively.  The proportion of male bill payers increases
from Thursday to Sunday {\bf (left)}, which is easier to read from
this display than from a stacked barchart {\bf (right)}.  }
\label{mosaic}
\end{figure}

\newpage
\subsection{Multivariate plots}~\label{pD}

\index{plots!multivariate}
\index{parallel coordinate plot}

%\subsubsection{Parallel coordinate plots for categorical or 
%real-valued variables}
\noindent{\bf Parallel coordinate plots for categorical or 
real-valued variables}
\bigskip

Parallel coordinate plots \cite{In85,We90} are frequently used for
examining multiple variables, looking for correlations, clustering in
high dimensions, and other relationships.  The plots are constructed
by laying out the axes in parallel instead of in the more familiar
orthogonal orientation of the Cartesian coordinate system. Cases are
represented by a line trace connecting the case value on each variable
axis.  Mathematician \citeasnoun{Oc1885} was the first to explain the
geometry of a parallel coordinate plot, and how it is that a point on
a graph of Cartesian coordinates transforms into a line in this other
space.

\begin{figure}[h]
\centerline{\includegraphics[width=4in]{chap-toolbox/par-crabs.pdf}}
\caption[Parallel coordinate plot of the \Data{Australian Crabs}
data]{Parallel coordinate plot of six of the seven variables of the
\Data{Australian Crabs} data, with males and females identified as
green rectangles and purple circles, respectively.  The relatively flat
traces indicate strong correlation between variables.}
\label{parallel}
\end{figure}


Figure~\ref{parallel} shows a parallel coordinate plot of one
categorical variable, \Vbl{sex}, and the five physical measurement
variables of the Blue \Vbl{species} of the \Data{Australian Crabs}.
The female crabs are shown by purple circles, and the male crabs are
shown by green rectangles.  We first note that the trace for each
crab is relatively flat, which indicates strong correlation between
variables.  (A small crab, for example, is small in all dimensions.)
Second, the lines for males and females cross between \Vbl{frontal
lobe}, \Vbl{rear width}, and \Vbl{carapace length}, which suggests that
the differences between males and females can be attributed to rear
width.

% We might transform (invert) a variable to minimize distracting
% edge crossings.

The order in which the parallel axes are positioned influences the
viewer's ability to detect structure.  Ordering the layout by a
numerical measure, such as correlation between variables, can be
helpful. When the variables have a natural order, such as time in
longitudinal or repeated measures, these plots are essentially the
same as profile plots. Parallel coordinate plots are also the same as
interaction plots, which are used in plotting experimental data
containing several factors.

%\bigskip
%\noindent {\bf Icon plots} In icon plots, one case equals one
%icon. Features of the icon are used to code variable values for the
%case. An example is the starplot \ref{***}.

% Maybe we don't need to discuss every possible plot, but list
% at the end of this chapter some of the plots we find useful but
% won't use in this book.   

\subsubsection{Scatterplot matrix, for categorical or 
real-valued variables}

\index{scatterplot matrix}

% multilayout is an odd word, and we don't discuss that word
% or the concept behind it anywhere in this chapter.

\begin{figure}[htp]
\centerline{\includegraphics[width=4in]{chap-toolbox/scatmat-bluecrabs.pdf}}
\caption[Scatterplot matrix of the real-valued variables in
\Data{Australian Crabs}]{Scatterplot matrix of the five real-valued
variables in \Data{Australian Crabs}.  The scatterplot matrix is a
display of related plots, in an arrangement that allows us to learn
from seeing them in relation to one another.  All five variables are
strongly linearly related, but some structure can be observed.}
\label{matrix}
\end{figure}

% Why use a scatterplot matrix?

The scatterplot matrix (draftsman's plot) contains pairwise
scatterplots of the $p$ variables laid out in a matrix format.  It is
a compact method for displaying a number of relationships at the same
time, and it offers something more, because this sensible plot
arrangement allows us to simultaneously make comparisons among all the
plots.  As with parallel coordinate plots, it is often useful to
re-order variables to highlight comparisons.

Figure~\ref{matrix} displays a scatterplot matrix of the five physical
measurement variables of the males and females for the Blue
\Vbl{species} of the \Data{Australian Crabs}, and the diagonal
displays the univariate ASH plot for each variable.  (Once again, male
and female crabs are represented by green rectangles and purple
circles, respectively.)  We can quickly see that all five variables are
strongly linearly related.  Note too that the two sexes differ in rear
width more than they do on the other variables; females have a
relatively larger value for \Vbl{rear width} than males.  The
difference is more pronounced in larger crabs.

It is instructive to compare the scatterplot matrix to the
correlation or covariance matrix for the same variables since they
share the same square structure.  For example, each plot in the
scatterplot matrix just discussed corresponds to a single number
in this correlation matrix:

\smallskip
\begin{center}
\begin{tabular}{l@{\hspace{.3em}}|@{\hspace{.3em}}ccccc}
&     FL \T \B &  RW &  CL &  CW &  BD\\
\hline
FL \T & 1.00 & 0.90 & 1.00 & 1.00 & 0.99\\
RW & 0.90 & 1.00 & 0.90 & 0.90 & 0.90\\
CL & 1.00 & 0.90 & 1.00 & 1.00 & 0.99\\
CW & 1.00 & 0.90 & 1.00 & 1.00 & 0.99\\
BD \B & 0.99 & 0.90 & 1.00 & 1.00 & 1.00
\end{tabular}
\end{center}
\smallskip

\noindent Just as one scatterplot contains a great deal more
information than can be captured by any one statistic, a scatterplot
matrix contains much more information than this table of statistics.

\subsubsection{Tours for real-valued variables}~\label{tours}

A tour \index{tour} is a motion graphic designed to study the joint
distribution of multivariate data \cite{As85}, in search of
relationships that may involve several variables. It is created by
generating a sequence of low-dimensional projections of
high-dimensional data; these projections are typically 1D or 2D.
Tours are thus used to find interesting projections of the data that
are not orthogonal, unlike those plotted in a scatterplot matrix and
other displays of marginal distributions.

Here is a numerical calculation illustrating the process of
calculating a data projection. For the five physical measurements in
the crabs data, in Table \ref{crabs-data}, if the 1D projection vector
is

\[
\blA_1 = \left[ \begin{array}{c} 1 \\ 0 \\ 0 \\ 0 \\ 0  \end{array} \right], 
~~~ \mbox{then the data projection is}~~~~ \blX\blA_1 = \left[ \begin{array}{c}  8.1 \\ 8.8 \\  9.2 \\ 9.6 \\ 7.2 \\  
   \vdots \end{array} \right],
\]


\noindent which is equivalent to the first variable, \Vbl{frontal lobe}.
Alternatively, if the projection vector is

\[
\blA_2 = \left[ \begin{array}{c} 0.707 \\ 0.707 \\ 0 \\ 0 \\ 0  \end{array} \right], 
~ \mbox{then } \blX\blA_2 = \left[
\begin{array}{c} 0.707\times 8.1 + 0.707\times 6.7=10.5 \\ 0.707\times 8.8
+0.707\times 7.7=11.7 \\ 0.707\times 9.2 + 0.707\times 7.8=12.0 \\ 0.707\times
9.6 + 0.707\times 7.9=12.4\\ 0.707\times 7.2 + 0.707\times 6.5 =9.7\\
\vdots
\end{array} \right],
\]

\noindent and the resulting vector is a linear combination of the
first two variables, \Vbl{frontal lobe} and \Vbl{rear width}.

The corresponding histograms of the full set of crabs data
are shown in Fig.~\ref{tour1D-examples}, with the projection of the
data onto $\blA_1$ at left and the projection onto $\blA_2$ at right.
%The projection of the data into $\blA_1$ is shown at left, and
%$\blA_2$ is shown at right. 
In the linear combination of two variables, $\blX\blA_2$, a suggestion
of bimodality is visible. The data is also more spread in this
projection: The variance of the combination of variables is
larger than the variance of \Vbl{frontal lobe} alone.

\begin{figure}[htp]
\centerline{
  {\includegraphics[width=2.3in]{chap-toolbox/tour-example1h.pdf}}
  {\includegraphics[width=2.3in]{chap-toolbox/tour-example2h.pdf}}
}
\caption[One-dimensional projections of the \Data{Crabs}
data]{Two 1D projections of the \Data{Crabs} data.  The
plotted vectors are different linear combinations of \Vbl{frontal
lobe} and \Vbl{rear width}, such that the first plot shows
\Vbl{frontal lobe} alone and the second shows an equal-weighted
projection of both variables.}
\label{tour1D-examples}
\end{figure}

Next, we will look at a numerical example of a 2D projection.  If the
2D projection matrix is

\[
\blA_3 = \left[ \begin{array}{cc} 1 & 0\\ 0 & 1\\ 0 & 0\\ 0 & 0\\ 0 & 0 
\end{array} \right],
~~~ \mbox{then the data projection is}~~~~ \blX\blA_3 = \left[
\begin{array}{cc} 8.1 & 6.7 \\ 8.8  & 7.7 \\ 9.2 & 7.8 \\ 9.6 & 7.9 \\ 
7.2 & 6.5 \\ \vdots \end{array} \right] ,
\]

\noindent which is equivalent to the first two variables, \Vbl{frontal
lobe} and \Vbl{rear width}.  Alternatively, if the projection matrix
is

\begin{eqnarray*}
& & \blA_4 = \left[ \begin{array}{cc} 0 & 0\\ 0 & 0.950\\ 0 & -0.312\\ 
  -0.312 & 0\\ 0.950 & 0 
\end{array} \right] 
~~~ \mbox{then }~~~~ \blX\blA_4 = \\
& &\left[
\begin{array}{cc} -0.312\times 19.0+0.950\times 7.0 = 0.72 & 
  ~~~0.950\times 6.7-0.312\times 16.1 = 1.34 \\ 
-0.312\times 20.8+0.950\times 7.4 = 0.54  & 
  ~~~0.950\times 7.7-0.312\times 18.1 = 1.67 \\ 
-0.312\times 22.4+0.950\times 7.7 = 0.33 & 
  ~~~0.950\times 7.8-0.312\times 19.0 = 1.48 \\
-0.312\times 23.1+0.950\times 8.2 = 0.58 & 
  ~~~0.950\times 7.9-0.312\times 20.1 = 1.23 \\ 
-0.312\times 17.1+0.950\times 6.1 = 0.46 & 
  ~~~0.950\times 6.5-0.312\times 14.7 = 1.59 \\ 
 \vdots \end{array} \right].
\end{eqnarray*}

\begin{figure}[htbp]
\centerline{
 {\includegraphics[width=2.3in]{chap-toolbox/tour-example3.pdf}}
 {\includegraphics[width=2.3in]{chap-toolbox/tour-example4.pdf}}}
\caption[Two-dimensional projections of the \Data{Crabs} data]{Two 2D
projections of the \Data{Crabs} data.  The first projection shows a
strong linear relationship, and the second projection shows clustering
within a narrower range of values.}
\label{tour2D-examples}
\end{figure}

\noindent The resulting matrix has two columns, the first of which is
a linear combination of \Vbl{carapace width} and \Vbl{body depth}, and
the second of which is a linear combination of \Vbl{rear width} and
\Vbl{carapace length}.

The corresponding bivariate scatterplots of the full set of crabs data
are shown in Fig.~\ref{tour2D-examples}. The projection of the data
into $\blA_3$ is shown at left, and the projection into $\blA_4$ is on
the right. The left-hand projection shows shows a strong linear
relationship, with larger variation among the high values. In the
second projection the points loosely clump into the four sectors of
the plot, and by inspecting the axes we can see that the variation in
values is much smaller in this projection.

The values in a projection matrix, $\blA$, can be any values on
$[-1,1]$ with the constraints that the squared values for each column
sum to 1 (normalization) and the inner product of two columns sums to
0 (orthogonality).

A tour \index{tour} is constructed by creating a sequence of
projection matrices, $\blA_t$. The sequence should be dense in the
space, so that all possible low-dimensional projections are equally
likely to be chosen in the shortest amount of time. Between each pair
of projections in the sequence an interpolation is computed to produce
a sequence of intermediate projection matrices ~---~ the result is a
sense of continuous motion when the data projections are
displayed. Watching the sequence is like watching a movie, except that
we usually watch in real time rather than watching a stored
animation. The algorithm is described in detail in
\citeasnoun{BCAH05}, and \citeasnoun{CLBW06}.

An alternative to motion graphics is available if the projection
dimension is 1.  In that case, the projections can be drawn as tour
curves, similar to Andrews curves \cite{An72}.


\bigskip
\noindent{\em How do we choose the projections to plot?} 
There are three methods for choosing projections, and they might
all be used at different points in an analysis:

%This looks a bit odd here to have a bulleted list and then followed immediately by
%longer descriptions of the three methods. Was this originally a paragraph.
\begin{itemize}
\item The grand tour, \index{tour!grand} in which the sequence of
  projections is chosen randomly.  It is designed to cover the space
  efficiently, and it is used to get acquainted with the data.
\item The projection pursuit guided tour, \index{tour!projection
pursuit guided} in which the sequence is guided by an algorithm in
search of ``interesting'' projections.  It is used to aid in the
search for specific patterns such as clusters or outliers, depending
on the projection pursuit function chosen by the user.
\item Manual manipulation, \index{tour!manual} in which the user
  chooses the sequence.  It is often used to explore the neighborhood
  around a projection that has been reached using a random or guided
  tour, perhaps in search of an even better projection.
\end{itemize}

\bigskip
\noindent {\em Grand tour:} In the grand tour, \index{tour!grand} 
the default method in GGobi, a random sequence of projections is
displayed.  It may be considered an interpolated random walk over the
space of all projections. This method used here was discussed
originally in \citeasnoun{As85} and \citeasnoun{BA86b}. Related work
on tours can be found in \citeasnoun{We91}, \citeasnoun{Ti91}, and
\citeasnoun{WPS98}.

Random projections can be generated efficiently by sampling from a
multivariate normal distribution. For a 1D projection vector having
$p$ elements, we start by sampling $p$ values from a standard
univariate normal distribution; the resulting vector is a sample from
a standard multivariate normal.  We standardize this vector to have
length 1, and the result is a random value on a $(p-1)$D sphere: This
is our projection vector.  For a 2D random projection of $p$
variables, follow this procedure twice, and orthonormalize the second
vector on the first one.

\bigskip
\index{tour!projection pursuit guided}
\noindent {\em Guided tour:} In a projection pursuit guided tour 
\cite{CBCH95}, the next target basis is selected by optimizing a
function that specifies some pattern or feature of interest; that is,
the sequence of projections, in the high-dimensional data space, is
moved toward low-dimensional projections that have more of this
feature.

\index{projection pursuit} Using projection pursuit (PP) in a static
context, such as in R or S-PLUS, yields a number of static plots of
projections that are deemed interesting.  Combining projection pursuit
with motion offers the interesting views in the context of surrounding
views, which allows better structure detection and interpretation.

\index{projection pursuit!indexes}
It works by optimizing a criterion function, called the projection
pursuit index, over all possible $d$D projections of $p$D data,

\[
\mbox{max} f(\blX\blA) ~~~\forall \blA,
\]

\noindent which are subject to the orthonormality constraints on $\blA$.

%Most projection pursuit indexes (for example, \cite{JS87};
%\cite{Fr87}; \cite{Ha89}; \cite{Mo89}; \cite{CBC93}; \cite{Po94}) have
%been anchored on the premise that to find the structured projections
%one should search for the most non-normal projections. Good arguments
%for this can be found in \citeasnoun{Hu85} and \cite{DF84}). (We
%should point out that searching for the most non-normal directions is
%also discussed by \citeasnoun{AGW71} in the context of transformations
%to enhance normality of multivariate data.)  This clarity of purpose
%makes it relatively simple to construct indexes which ``measure'' how
%distant a density estimate of the projected data is from a standard
%normal density.  The projection pursuit index, a function of all
%possible projections of the data, invariably has many ``hills and
%valleys'' and ``knife-edge ridges'' because of the varying shape of
%underlying density estimates from one projection to the next.

The projection pursuit indexes in our toolbox include holes, central
mass, LDA, and PCA (1D only). For an $n\times d$ matrix of projected
data, $\bly=\blX\blA$, these indexes are defined as follows:

\begin{itemize}
\item[Holes]:

\[
I_{\rm holes}(\blA)= \frac{1-\frac{1}{n}\sum_{i=1}^{n}\exp(-\frac{1}{2}\bly_i\bly_i')}{1-\exp(-\frac{p}{2})}
\]
\index{projection pursuit!indexes!holes}

\item[Central mass]:

\[
I_{\rm CM}(\blA)= \frac{\frac{1}{n}\sum_{i=1}^{n}\exp(-\frac{1}{2}\bly_i\bly_i')-\exp(-\frac{p}{2})}{1-\exp(-\frac{p}{2})}
\]
\index{projection pursuit!indexes!central mass}

\item[PCA]: ~~~~Defined for 1D projections, $d=1$.

\[
I_{\rm PCA} = \frac{1}{n} \sum_{i=1}^n \bly_i^2
\]
\index{projection pursuit!indexes!PCA}

\newpage
\item[LDA]:

\[
I_{\rm LDA}(\blA) = 1- \frac{|\blA'\blW\blA|}{|\blA'(\blW+\blB)\blA|}
\]
\index{projection pursuit!indexes!LDA}

\noindent 
where $\blB = \sum_{i=1}^g
n_i(\bar{\bly}_{i.}-\bar{\bly}_{..})(\bar{\bly}_{i.}-\bar{\bly}_{..})',
\blW=\sum_{i=1}^g\sum_{j=1}^{n_i}
(\bly_{ij}-\bar{\bly}_{i.})(\bly_{ij}-\bar{\bly}_{i.})'$ are the
between- and within-group sum of squares matrices in a linear
discriminant analysis, with $g=$number of groups, and $n_i, i=1, ....
g$ is the number of cases in each group. 

\end{itemize}

% Sphering
\index{sphering} These formulas are written assuming that $\blX$ has 
a mean vector equal to zero and a variance--covariance matrix equal to
the identity matrix; that is, it has been sphered (transformed into
principal component scores) or equivalently transformed into principal
component coordinates. An implementation can incorporate these
transformations on the fly, so sphering the data ahead of PP is not
necessary. However, in our experience all of the projection pursuit
indexes find interesting projections more easily when the data is
sphered first.
\index{principal component analysis}

The holes and central mass indexes \cite{CBC93} derive from the normal
density function. The first is sensitive to projections with few
points in the center of the projection, and the second to those with a
lot of points in the center.  The LDA index \cite{LCKL04} derives from
the statistics for MANOVA, and it is maximized when the centers of
pre-defined groups in the data (indicated by symbol) are farthest
apart. The PCA index derives from principal component analysis, and it
finds projections where the data is most spread. Figures~\ref{pp1} and
\ref{pp2} show a few projections found using projection pursuit guided
tours on the \Data{Crabs} data.

\begin{figure}[htp]
\centerline{{\includegraphics[width=2in]{chap-toolbox/tour-holes2.pdf}}
 {\includegraphics[width=2in]{chap-toolbox/tour-holes1.pdf}}}
\smallskip
\centerline{{\includegraphics[width=2in]{chap-toolbox/tour-mass.pdf}}
 {\includegraphics[width=2in]{chap-toolbox/tour-lda.pdf}}}
\caption[Two-dimensional projections found by projection
pursuit]{Two-dimensional projections found by projection pursuit on
the \Data{Crabs} data. Two projections from the holes index {\bf (top
row)} show separation between the four classes.  A projection from the
central mass index {\bf (bottom left)} shows the concentration of points
in the center of the plot.  A projection from the LDA index {\bf (bottom
right)} reveals the four classes.}
\label{pp1}
\end{figure}

\begin{figure}[htp]
\centerline{{\includegraphics[width=2in]{chap-toolbox/tour1d-holes.pdf}}
 {\includegraphics[width=2in]{chap-toolbox/tour1d-mass.pdf}}}
\smallskip
\centerline{{\includegraphics[width=2in]{chap-toolbox/tour1d-pca.pdf}}
 {\includegraphics[width=2in]{chap-toolbox/tour1d-lda.pdf}}}
\caption[One-dimensional projections found by projection
pursuit]{One-dimensional projections found by projection pursuit on
the \Data{Crabs} data. A projection from the holes index has found
separation by \Vbl{species} {\bf (top left)}.  A projection from the
central mass index shows a density with short tails {\bf (top right)};
not especially useful for this data).  The PCA index {\bf (bottom left)}
result shows bimodality but no separation.  A projection from the LDA
index {\bf (bottom right)} reveals the species separation.  }
\label{pp2}
\end{figure}

% Optimization algorithm
Optimizing the projection pursuit index is done by a derivative-free
random search method. A new projection is generated randomly. If the
projection pursuit index value is larger than the current value, the
tour moves to this projection. A smaller neighborhood is searched in
the next iteration. The possible neighborhood of new projections
continues to shrink, until no new projection can be found where the
projection pursuit index value is higher than that of the current
projection.  At that point, the tour stops at that local maximum of
the projection pursuit index. To continue touring, the user needs to
break out of the optimization, either reverting to a grand tour or
choosing a new random start.


% Add algorithm on manual controls
\bigskip
\noindent {\em Manual tour:} \index{tour!manual} In order to explore
the immediate neighborhood around a particular projection, we pause
the automatic motion of the tour and go to work using the mouse.  One
variable is designated as the ``manipulation variable,'' and the
projection coefficient for this variable is controlled by mouse
movements; the transformation of the mouse movements is subject to the
orthonormality constraints of the projection matrix. In GGobi, manual
control is available for 1D and 2D tours.

In 1D tours the manipulation variable is rotated into or out of the
current 1D projection.

In 2D tours, GGobi offers unconstrained oblique manipulation or
constrained manipulation; if the manipulation is constrained, the
permitted motion may be horizontal, vertical, or angular.  With only
three variables, the manual control works like a trackball: There are
three rigid, orthonormal axes, and the projection is rotated by
pulling the lever belonging to ``manip'' variable.  

With four or more variables, a 3D manipulation space is created from
the current 2D projection augmented by a third axis controlled by the
``manip'' variable. This manipulation space results from
orthonormalizing the current projection with the third axis.  When the
``manip'' variable is dragged around using the mouse, its coefficients
follow the manual motion, and the coefficients of the other variables
are adjusted accordingly, such that their contributions to the 3D
manipulation space are maintained.

When using the manual tour, we experiment with the choice of
manipulation variable.  We pause the tour to explore some interesting
structure, and we select a manipulation variable.  We may select a
variable at random or make our selection based on our understanding
of the data.  We vary the variable's contribution to the projection
and watch what happens, assessing the sensitivity of the structure to
that variation.  If we are lucky, we may be able to sharpen or
refine a structure first exposed by the grand or guided tour.

%\begin{figure}[htp]
%%\vspace{-1.5in}
%\centerline{\includegraphics[width= 2in]{chap-toolbox/globe-annot.pdf}\includegraphics[width=2in]{chap-toolbox/globe2-annot.pdf}\includegraphics[width=2in]{chap-toolbox/globe4-annot.pdf}}
%\caption{A schematic picture of trackball controls. The semblance of a globe 
%is rotated by manipulating the contribution of $X_1$ in the projection.}
%\label{globe}
%\end{figure}

%\begin{figure}[htp]
%\centerline{\includegraphics[width=3in]{chap-toolbox/4D-basis1-annot.pdf}\includegraphics[width=3in]{chap-toolbox/4D-basis2-annot.pdf}}
%\caption{Constructing the 3-dimensional manipulation space to
%manipulate the contribution of variable 1 in the projection.}
%\label{manip-space}
%\end{figure}

%\begin{figure}[htp]
%\centerline{\includegraphics[width=1.5in]{chap-toolbox/oblique.pdf}\hspace{-0.2in}\includegraphics[width=1.5in]{chap-toolbox/vert.pdf}\hspace{-0.2in}\includegraphics[width=1.5in]{chap-toolbox/horiz.pdf}\hspace{-0.2in}\includegraphics[width=1.5in]{chap-toolbox/radial.pdf}\hspace{-0.2in}\includegraphics[width=1.5in]{chap-toolbox/angular.pdf}}
%\caption{Two-dimensional variable manipulation modes: dashed line represents variable contribution to the projection before manipulation and the solid line is the contribution after manipulation.}
%\label{control-types}
%\end{figure}

%The types of manual control available in GGobi are ``oblique'', adjust
%the variable contribution in any direction, ``horizontal'', allow
%adjustments only in the horizontal plot direction, ``vertical'', allow
%adjustments only in the vertical plot direction, ``radial'', allow
%adjustments only in the current direction of contribution,
%``angular'', rotate the contribution within the viewing plane. These
%are demonstrated in Fig.~\ref{control-types}.

\bigskip
\noindent{\em Relationships between tours and numerical algorithms:}

The tour algorithm is a cousin to several numerical algorithms often
used in statistical data analysis ~---~ algorithms that generate
interesting linear combinations of data. These linear combinations
might be fed into a tour, or re-created directly using manual
manipulation, after which their neighborhoods can be explored to see
whether some nearby projection is even more informative.

\index{principal component analysis}

In principal component analysis, for example, the principal component
is defined to be $\blY = (\blX-\bar{\blX})\blA$, where $\blA$ is the
matrix of eigenvectors from the eigen-decomposition of the
variance--covariance matrix of the data, $\blS =
\blA\Lambda\blA'$. Thus a principal component is one linear projection
of the data and could be one of the projections shown by a tour.  A
biplot \cite{Ga71,GH96} is a scatterplot that is similar to some 2D
tour views: Plot the first two principal components, and then add the
coordinate axes, which are analogous to the ``axis tree'' that is
added at the lower left corner of 2D tour plots in GGobi.  These axes
are used to interpret any structure visible in the plot in relation to
the original variables. Figure~\ref{biplot} shows biplots of the
\Data{Australian crabs} above a tour plot showing a similar
projection, which is constructed using manual controls.

\begin{figure}[htp]
\centerline{\includegraphics[width=2.3in]{chap-toolbox/biplot1.pdf}
  \includegraphics[width=2.3in]{chap-toolbox/biplot2.pdf}}
\centerline{\includegraphics[width=2in]{chap-toolbox/biplot3.pdf}}
\caption[The relationship between a 2D tour and the biplot]{The
relationship between a 2D tour and the biplot: a biplot {\bf (top
left)} of the five physical measurement variables of the
\Data{Australian Crabs}; the same plot scaled to equal length axes
{\bf (top right)}; the same projection {\bf (bottom)} produced using
the manually controlled tour.}
\label{biplot}
\end{figure}

In linear discriminant analysis, the Fisher linear discriminant (the
linear combination of the variables that gives the best separation of
the class means with respect the class covariance) is one projection
of the data that may be shown in a tour.  In support vector machines
\cite{Va99}, projecting the data into the normal to the separating
hyperplane yields another.  Canonical correlation analysis and
multivariate regression also generate interesting projections; these
may be explored in a 2$\times$1D tour.
\index{regression}
\index{canonical correlation analysis}
\index{classification methods!linear discriminant analysis (LDA)}
\index{classification methods!support vector machines}

\subsection{Plot arrangement}

\index{plots!arrangement}

Viewing multiple plots simultaneously is a good way to make
comparisons among several variables.  Examples include scatterplot
matrices, parallel coordinate plots, and trellis plots
\cite{BeCS96}. In each of these, several plots are arranged in a
structured manner.

\index{scatterplot matrix}
\index{parallel coordinate plot}

A scatterplot matrix lays out scatterplots of all pairs of the
variables in a format matching a correlation or covariance matrix,
which allows us to examine 2D marginal distributions in relation to
one another. A parallel coordinate plot lays out all univariate
dot plots in a sequential manner, which shows all 1D marginal
distributions. The connecting lines give us more information about the
relationships between variables, the joint distribution.

\index{trellis plot} The arrangement of a trellis plot is different.
The data is usually partitioned into different subsets of the cases,
with each subset plotted separately. The partitioning is commonly
based on the values of a categorical variable or on a binned
real-valued variable. In statistical terms, we use trellis plots to
examine the conditional distributions of the data. Brushing,
\index{brushing} which is described in the next section (\ref{brush}),
can provide similar information.

\section{Plot manipulation and enhancement}

\subsection{Brushing}~\label{brush}

Brushing works by using the mouse to control a ``paintbrush,''
directly changing the color or glyph of elements of a
plot. \index{brushing} The paintbrush \index{paintbrush} is sometimes
a pointer and sometimes works by drawing an outline of sorts around
points; the outline is sometimes irregularly shaped, like a lasso.  In
GGobi, the brush is rectangular when it is used to brush points in
scatterplots or rectangles in area plots, and it forms a crosshair
when it is used to brush lines.

%Automatic brushing from R?  or from the GUI?
\newpage
Brushing is most commonly used when multiple plots are visible and
some linking mechanism exists between the plots. There are several
different conceptual models for brushing and a number of common
linking mechanisms. 
\index{brushing!linked}

\subsubsection{Brushing as database query}
\index{brushing!database query}

We can think about linked brushing \index{brushing!linked} as a query
of the data.  Two or more plots showing different views of the same
dataset are near one another on the computer screen.  A user poses a
query graphically by brushing a group of plot elements in one of the
plots. The response is presented graphically as well, as the
corresponding plot elements in linked plots are modified
simultaneously.  A graphical user interface that supports this
behavior is an implementation of the notion of ``multiple linked
views,'' which is discussed in \citeasnoun{AWS91} and
\citeasnoun{BMMS91}.

\index{multiple views}

To illustrate, we set up two pairwise scatterplots of data on
\Data{Australian Crabs}, \Vbl{frontal lobe} vs. \Vbl{rear width}
and \Vbl{sex} vs. \Vbl{species}. (In the latter plot, both
variables are jittered.)  These plots, shown in Fig.~\ref{query}, are
linked, case to case, point to point.  We brush the points at the
upper left in the scatterplot of \Vbl{sex} vs. \Vbl{species},
female crabs of the Blue \Vbl{species}, and we see where those points
appear in the plot of \Vbl{frontal lobe} vs \Vbl{rear width}.  If we
were instead working with the same data in a relational table in a
database, we might pose the same question using SQL.  We would issue
the following simple SQL query and examine the table of values of
\Vbl{frontal lobe} and \Vbl{rear width} that it returns:

\begin{verbatim}
SELECT frontal_lobe, rear_width FROM Australian_Crabs
WHERE sex = female AND species = Blue
\end{verbatim}

\begin{figure}[htp]
\centerline{
  \includegraphics[width=2in]{chap-toolbox/brush-query.pdf}
  \includegraphics[width=2in]{chap-toolbox/brush-response.pdf}
}
\caption[A dynamic query posed using linked brushing]{A dynamic query
posed using linked brushing.  The points corresponding to \Vbl{sex} =
female and \Vbl{species} = Blue are highlighted.}
\label{query}
\end{figure}

Brushing is a remarkably efficient and effective way to both pose the
query \index{brushing} and view the result, allowing easy comparison
of the values in the resulting table to the remaining values.  Note
that the plots containing query and response should be placed close
together, so that the action occurs in the same visual neighborhood.

You may have already realized that the same task can be performed
using a single window by first marking the queried objects and then
selecting new variables to plot.  Indeed, that approach has its uses,
but it is much slower and the user may lose context as the view
changes. Having several views available simultaneously allows the
information to be rapidly absorbed.

A limitation of replacing SQL with brushing is that it may be
difficult to form clauses involving many variables or otherwise
complicated selections.  Because most brushing interfaces use an
accumulating mode of painting, it is easy to form disjunctions
(unions) of any kind, but it may be unintuitive to form conjunctions
(intersections) involving many variables. For such queries, a command
line interface may be more efficient than extensive graphical control.

\subsubsection{Brushing as conditioning}
\index{brushing!conditioning}

Since brushing \index{brushing} a scatterplot defines a clause of the
form $(X,Y) \in B$, the selection can be interpreted as conditioning
on the variables $X$ and $Y$.  As we vary the size and position of the
area $B$, we examine the dependence of variables plotted in the linked
windows on the ``independent'' variables $X$ and $Y$. Thus linked
brushing allows the viewer to examine the conditional distributions in
a multivariate dataset, and we can say that Fig.~\ref{query} shows
the conditional distribution of \Vbl{frontal lobe} and \Vbl{rear
width} on $X =$
\Vbl{species} and $Y = $ \Vbl{sex}.

This idea underlies scatterplot brushing described in
\citeasnoun{BC87} and forms the basis for trellis plots 
\cite{BeCS96}.

\subsubsection{Brushing as geometric sectioning} 
\index{brushing!geometric sectioning}

Brushing projections \index{brushing} with thin point-like or
line-like brushes can be seen as forming geometric sections with
hyperplanes in data space \cite{FB94}.  From such sections the true
dimensionality of the data can be inferred under certain conditions.

\subsubsection{Linking mechanisms}

% Example 1 -- case to case

Figure~\ref{query} shows the simplest kind of linking, one-to-one,
where both plots show different projections of the same data, and a
point in one plot corresponds to exactly one point in the other.  When
using area plots, brushing any part of an area has the same effect as
brushing it all and is equivalent to selecting all cases in the
corresponding category.  Even when some plot elements represent more
than one case, the underlying linking rule still links one case in one
plot to the same case in other plots.  \index{linking!one-to-one}

It is possible, though, to define a number of different linking rules.
Your dataset may include more than one table of different
dimensions, which are nevertheless related to one another through one
or more of their variables.  For example, suppose your dataset on
crabs included a second table of measurements taken on Alaskan crabs.
Instead of 200 crabs, as the first table describes, suppose the second
table had data on only 60 crabs, with a few ~---~ but not all ~---~ of
the same variables.  You still might like to be able to link those
datasets and brush on \Vbl{sex} so that you could compare the two
kinds of crabs.  To establish that link, some rule is needed that
allows $m$ points in one plot to be linked to $n$ points in another.
\index{linking!$m$-to-$n$}

% Example 2 -- ``categorical'' brushing

\begin{figure}[htp]
\centerline{\includegraphics[width=2.5in]{chap-toolbox/catbrush.pdf}}
\caption[Categorical brushing]{Categorical brushing.  Only a single
point in the scatterplot of \Vbl{frontal lobe} vs. \Vbl{rear width} is
enclosed by the brush, but because the linking rule specifies linking
by \Vbl{sex}, all points with \Vbl{sex} = female have been brushed.}
\label{catbrush}
\end{figure}

Figure~\ref{catbrush} shows a variation we call linking by variable,
or categorical brushing, \index{linking!categorical} in which a
categorical variable is used to define the linking rule.  It is used
in plots of real-valued variables when the data also includes one or
more categorical variables.  Only a single point in the scatterplot is
enclosed by the brush, but because the linking rule specifies linking
by \Vbl{sex}, all points with the same level of \Vbl{sex} as the
brushed point have been brushed simultaneously.  In this simple
example, $m=n$ and this takes place within a single plot.  When we
link two matrices of different dimension that share a categorical
variable, we have true $m$ to $n$ linking.

% For area plots, this distinction appears to be moot. !!

Note that this distinction, linking by case or by variable, is moot
for area plots.  When the brush intersects an area, the entire area is
brushed, so all cases with the corresponding levels are brushed ~---~
in that plot and in any linked plots.  Linking by variable enables us
to achieve the same effect within point plots.

Figure~\ref{mtonlinking} illustrates another example of $m$-to-$n$
linking.  The data (\Data{Wages}) contains 6,402 longitudinal
measurements for 888 subjects.  We treat the subject identifier as a
categorical variable, and we specify it as our linking variable.
% so $m=1$ and $n$ is equal to the
% number of time units for each individual.  
When we brush any one of a subject's points, all other points
corresponding to this subject are simultaneously highlighted ~---~ and
in this instance, all connecting line segments between those points
as well.  In this case, we link $m$ points in the dataset with $n$
edges.

% I don't think we need this because of my hypothetical example
% above about the Australian and Alaskan crabs.  dfs
%Note that the plots don't need to represent the same data matrix.  The
%dataset can include more than one table, and they can be linked
%according to a common categorical variable.

\begin{figure}[htp]
\centerline{
  \includegraphics[width=2.0in]{chap-toolbox/wages1.pdf}
  %\includegraphics[width=1.5in]{chap-toolbox/wages2.pdf}
  \includegraphics[width=2.0in]{chap-toolbox/wages3.pdf}}
\caption[Linking by subject identifier]  {Linking by subject
identifier in longitudinal data, \Data{Wages}, for two different
subjects.  The linking rule specifies linking by subject ID, so
brushing one point causes all $m$ points and $n$ edges for that
subject to be highlighted. }
\label{mtonlinking}
\end{figure}

% Example 3 -- point to edge

\begin{figure}[htp]
\begin{center}
\includegraphics[width=2in]{chap-toolbox/rmalink1.pdf}
\includegraphics[width=2in]{chap-toolbox/rmalink2.pdf}
%\centerline{{\pdfimage width 2in{chap-toolbox/linking1.jpg}}{\pdfimage width 2in{chap-toolbox/linking2.jpg}}}
\end{center}
\caption[Linking between a point in one plot and a line in another,
using \Data{Arabidopsis Gene Expression}]{Linking between a point in
one plot and a line in another, using \Data{Arabidopsis Gene
Expression}. The line segments in the right-hand plot correspond
exactly to the points in the left-hand plot.}
\label{linking}
\end{figure}

As we have just illustrated, we can link different kinds of objects.
We can brush a point in a scatterplot that is linked to an edge in a
graph. \index{linking!point-to-edge} Figure~\ref{linking} illustrates
this using the \Data{Aradopsis Gene Expression} data. The left plot
shows the $p$-values vs. the mean square values from factor 1 in an
analysis of aariance (ANOVA) model; it contains 8,297 points.
The highlighted points are cases that have small $p$-values but large
mean square values; that is, there is a lot of variation, but most of
it is due to the treatment. The right plot contains 16,594 points
that are paired and connected by 8,297 line segments. One line
segment in this plot corresponds to a point in the left-hand
scatterplot.

\subsubsection{Persistent vs. transient --- painting vs. brushing}

% transient vs persistent, brushing vs painting

\begin{figure}[htp]
\hbox{\includegraphics[width=1.5in]{chap-toolbox/brushing1.pdf}
      \includegraphics[width=1.5in]{chap-toolbox/brushing2.pdf}
      \includegraphics[width=1.5in]{chap-toolbox/brushing3.pdf}}
\smallskip
\hbox{\includegraphics[width=1.5in]{chap-toolbox/brushing4.pdf}
      \includegraphics[width=1.5in]{chap-toolbox/brushing5.pdf}
      \includegraphics[width=1.5in]{chap-toolbox/brushing6.pdf}}
\caption[Transient brushing contrasted with persistent
painting]{Transient brushing contrasted with persistent painting.
A sequence of transient operations {\bf (top row)}, in which the
points return to their previous color when the brush moves off.  
In persistent painting {\bf (bottom row)}, points retain the new
color.  }
\label{brushing}
\end{figure}

% Skip this figure
%\begin{figure}[htp]
%\centerline{\hbox{\includegraphics[width=2in]{chap-toolbox/line-painting1.pdf}
%  \includegraphics[width=2in]{chap-toolbox/line-painting2.pdf}}}
%\caption{Persistently painting lines in a plot.  The two plots
%illustrate a sequence of brushing operations.}
%\label{line-brushing}
%\end{figure}

\index{brushing!transient}\index{brushing!persistent}\index{painting}
Brushing scatterplots can be a transient operation, in which points in
the active plot only retain their new characteristics while they are
enclosed or intersected by the brush, or it can be a persistent
operation, so that points retain their new appearance after the brush
has been moved away.  Transient brushing is usually chosen for linked
brushing, as we have just described.  Persistent brushing is useful
when we want to group the points into clusters and then proceed to use
other operations, such as the tour, to compare the groups. It is
becoming common terminology to call the persistent operation
``painting,'' and to reserve ``brushing'' to refer to the transient
operation.

Painting can be done automatically, by an algorithm, rather than
relying on mouse movements.  The automatic painting tool in GGobi will
map a color scale onto a variable, painting all points with the
click of a button.  (The GGobi manual includes a more detailed
explanation.)  Using the \RPackage{rggobi} package, the
\verb,setColors, and \verb,setGlyphs, commands can be used to paint
points.


\subsection{Identification}

\index{identification}
\index{identification!persistent}
\index{identification!transient}

Identification, which could also be called labeling or label brushing,
is another plot manipulation that can be linked.  Bringing the cursor
near a point or edge in a scatterplot, or a bar in a barchart, causes
a label to appear that identifies the plot element.  An
identification parameter can be set that changes the nature of the
label.  Figure~\ref{identify} illustrates different attributes shown
using identification in GGobi: row label, variable value, and record
identifier. The point highlighted is a female crab of the Orange
\Vbl{species}, with a value of 23.1 for \Vbl{frontal lobe}, and it is
the 200th row in the data matrix. Generally, identifying points is
best done as a transient operation, but labels can be made to stick
persistently. You will not be tempted to label many points persistently
at the same time, because the plot will be too busy.

\begin{figure}[htp]
\hbox{{\includegraphics[width=1.5in]{chap-toolbox/identify1.pdf}}
 {\includegraphics[width=1.5in]{chap-toolbox/identify2.pdf}}
 {\includegraphics[width=1.5in]{chap-toolbox/identify3.pdf}}}
\caption[Identifying points in a plot using three different labeling
styles]{Identifying points in a plot using three different labeling
styles.  The same point is labeled with its row label, variable value,
or record identifier.}
\label{identify}
\end{figure}

\subsection{Scaling}

\index{view scaling}

% zooming, its not drill down
% aspect ratio, cleveland's banking to 45deg
% explain data
% close to 1-1 global scale 
% + in this river example see the gap with the spring rain
% with banking, the seasonality emerges
% fixing scales of variables
% does multivariate scaling get mentioned here? or should it appear somewhere else?
% i think i've discussed this in the tour section, and in the par coords section
Scaling maps the data onto the window, and changes in that mapping
function help us learn different things from the same plot.  Scaling
is commonly used to zoom in on crowded regions of a scatterplot, and
it can also be used to change the aspect ratio of a plot, to reveal
different features of the data.

In the simple time series example shown in Fig.~\ref{scaling}, we
look at average monthly river flow plotted against time.  The left
plot shows the usual default scaling: a one-to-one aspect ratio.  We
have chosen this aspect ratio to look for global trends.  An eye for
detail might spot a weak trend, up and then down, and some
large vertical gaps between points at the middle top.

The vertical gap can be seen even more explicitly if we zoom in (as
you can try on your own), and there is a likely physical explanation: a
sudden increase in river flow due to spring melt, followed by an
equally sudden decline.

In the right plot, the aspect ratio has been changed by shrinking the
vertical scale, called \Term{banking to $45^o$} by \citeasnoun{Cl93}.  This
reveals the seasonal nature of the river flow more clearly, up
in the spring and down in the summer and fall.

As you can see, no single aspect ratio reveals all these
features in the data equally well.  This observation is consistent
with the philosophy of exploratory data analysis, as distinct from a
completely algorithmic or model-based approach, which may seek to
find the ``optimal'' aspect ratio.

\begin{figure}[htp]
\hbox{\includegraphics[width=2.3in]{chap-toolbox/scaling1.pdf}
      \includegraphics[width=2.3in]{chap-toolbox/scaling3.pdf}}
\caption[Scaling a plot reveals different aspects of the \Data{River
Flow} data]{Scaling a plot reveals different aspects of the
\Data{River Flow} data. At the default aspect ratio, a weak global
trend is visible, up then down, as well as vertical gaps at the
top. Flattening the plot highlights its seasonality.}
\label{scaling}
\end{figure}

%The above describes the view scaling manipulations a user can perform.
%Underlying the plot construction there is implicit scaling of the
%data. Typically, each plot is constructed using the minimum data value
%at the bottom and the maximum data value at the top of each axis.
%Sometimes we want a group of variables to be scaled together, and that
%can be arranged by overriding the default variable limits.  For tours,
%before projecting the data, we scale it into a high-dimensional
%rectangle using the limits for each variable.  These data pipeline
%methods are described in \citeasnoun{BAHM88} and
%\citeasnoun{SRLLDCC00}.

% I'm not sure about any of the following, but this gets something
% written down that we can work from.  I'm not sure when to include
% a description and when to refer people to the manual or the software.
% I don't want to say a lot about graphs yet when we don't even have
% a graph chapter -- another teaser?

\subsection{Subset selection}

Data analysts often partition or subset a data matrix and work with a
portion of the cases.  This strategy is helpful for several reasons.
Sometimes the number of records is simply too large, and it slows down
the computational software and causes a scatterplot to appear as an
undifferentiated blur.  If the data is fairly homogeneous, having no
obvious clusters, a random sample of cases can be selected for testing
and exploration.  We can sample the cases in R, before handing the
data to GGobi, or in GGobi itself, using the \Button{Subset} tool.

If the data is inhomogeneous, even when the number of cases is not
large, we may partition the cases in some structured way.  One common
partitioning strategy is to segment the data according to the levels
of a categorical variable, creating groups in the data that can be
analyzed and modeled one at a time.  In that case, we may partition
the cases in R before handing it to GGobi, or we may use brushing.  We
can paint the cases according to the levels of the partitioning
variable, and then use the \Button{Color and glyph groups} tool to
show only the group we are currently studying.

If no categorical variable is available, we may choose one or more
real-valued partitioning variables, dividing up the data according
to their values.

\subsection{Line segments}

Line segments have already been mentioned as an important element of
any display of longitudinal data.  We also use them to illustrate and
evaluate models, enhancing scatterplots with lines or grids, as you
will see often in the chapters to follow.  (See
Figs.~\ref{prim7-model} and \ref{clust-SOMb}.)

\index{graph}

In network analysis, the data includes a graph: a set of nodes
connected by edges.  In a social network, for example, a node often
represents a person, and an edge represents the connection between two
people.  This connection may have associated categorical variables
characterizing the connection (kinship, sexual contact, email
exchanges, etc.) and real-valued variables capturing the frequency or
intensity of the connection.  Using GGobi, we can lay out graphs in a
space of any dimension and use linked brushing to explore the values
corresponding to both nodes and edges.

\subsection{Interactive drawing}

\index{drawing}

We sometimes enhance plots of data by adding points and lines to
create illustrations of the model.  (See Fig.~\ref{prim7-tour}.)  Such
illustrations can be prepared in R, or by using a text editor to edit
the XML\textsuperscript{\textregistered} data file, or they can be
added interactively, using the mouse.  The same tools allow us to edit
graphs by hand.

\subsection{Dragging points}

\index{dragging points}

Moving points is a very useful supplement to graph layout: No layout
is perfect, and occasionally a point or group of points must be moved
in order to clarify a portion of the graph.  

You move a data point at your own peril, of course!

% I'll add this just to get us started --  it's a section that
% needs to be filled in.
% This will be a good place to cite some additional references
%   W05,unwin:2006a,haerdle;  Hofmann, Theus, Wegman, Unwin, Inselberg,
%    MacEachren, ...

\section{Tools available elsewhere}

Many valuable tools are not currently in the GGobi toolbox but
available in other software. Here we describe a few of these, and the
software where the reader can find them.

\begin{itemize}
% make not sound as though we're missing so much
% the rggobi connection surpasses any other problems
% maps, 
% alpha-blending - refer to large datasets book, and wegman, crystal vision
%   RGl, 
% parallel coordinate plots - winkler can do angle selections, ward, wegman
% glyphs - grinstein on maps, ward
% 

\index{MANET} \index{Mondrian}
\index{mosaic plot} \index{double decker plot}
\item {\em Mosaic plots:} When two or more categorical variables are
to be plotted, a good choice is a mosaic plot \cite{HK81} or
double-decker plot \cite{Ho01}. These displays recursively partition a
rectangle, with each rectangle representing the number of cases in the
corresponding category; a good description can be found in
\citeasnoun{Ho03}.  Direct manipulation includes the ability to manage
the nesting order of variables in multivariate mosaic plots, and to
interactively change the order in which the levels of categorical
variables are displayed. They are also efficient plots for large data,
since only a few rectangles may represent many thousands of points.
Graphics for large data are extensively discussed in
\citeasnoun{Unwin06} and rich implementations of mosaic plots are
included in MANET \cite{UHS96} and Mondrian \cite{Theus02}.

\index{R package!\RPackage{iPlots}}
\index{brushing!selection sequences}
\item {\em Selection sequences:} If only two colors are used in a
plot, one for background and one for highlighting, then brushing can
be extended using \Term{selection sequences} \cite{HT98}. One subset
selected by a brush can be combined with another brushed subset using
Boolean logic. This tool is available in MANET \cite{UHS96} and
Mondrian \cite{Theus02}.

\index{XmdvTool}
\item {\em Alpha blending:} When many cases are plotted in a
scatterplot or parallel coordinate plot, over-plotting is often a
problem.  One way to to see beneath the top point or line is to use
\Term{alpha blending}, where ink is laid down in transparent
quantities.  The more points drawn at the same location, the darker
the spot becomes. Alpha blending is discussed in \citeasnoun{CWL96},
and a variant is used in XmdvTool \cite{FWR99}.  Alpha levels can be
adjusted interactively in Mondrian and in the R package
\RPackage{iPlots} \cite{iPlots03}.
% I think the heavier spots are actually drawn lighter sometimes,
%  rather than darker.  dfs

%Should reference some of Carr and Wegman's stuff too, but
% don't want to reference crystalvision, and otherwise it looks like
% its just tech reports.

\item {\em Brushing lines in a parallel coordinate plot:} Brushing in
parallel coordinate plots is enhanced if it is possible to select
lines as well as points. This is discussed in \citeasnoun{CWL96} and
explored in CASSATT \cite{UVW03}, and it is supported in MANET
\cite{UHS96} and Mondrian \cite{Theus02}. XmdvTool also has good
support for parallel coordinate plots.

%\index{Chernoff faces} \index{star glyphs} \index{plots! icons}
%\item Icons (e.g. \cite{An57,Ch73} have been used to represent 
%multivariate data. 
%An icon plot maps ...

\index{GeoVista Studio} \index{ESTAT}
\item {\em Maps:} A lot of data arises in a spatial context, and it is
invaluable to show the map along with the data. See for example
\citeasnoun{AB97}, and \citeasnoun{CMSC95}.  GeoVista Studio
\cite{TaGa02} is a current development environment for creating
software for exploring geospatial data, and ESTAT is a tool it has
been used to build.  In addition to scatterplots and highly
interactive parallel coordinate plots, it includes linked displays of
maps.

\end{itemize}

%\item  Fully interactive parallel coordinate plots (Inselberg, Wegman?)
%\item  Treemaps (Schneiderman)
%\item  Alpha blending, 3D (Mondrian, iPlots, CrystalVision)
%\item  Maps (GeoVISTA Studio, ...)
% Mondrian: links to databases

\section{Recap}

In this short introduction to our toolbox, we have not described all
the tools in GGobi.  Others are documented in the manual on the
web site, which will change as the software evolves. The collection of
methods in the toolbox is a very powerful addition for exploratory
data analysis. We encourage the reader to explore the methods in
more depth than is covered in this book, develop new approaches using
\RPackage{rggobi}, and explore the tools available in other software.

\section*{Exercises}
\addcontentsline{toc}{section}{Exercises}

\begin{enumerate}

\item For the \Data{Flea Beetles}:
\begin{enumerate}
\item Generate a parallel coordinate plot of the six variables, with
the color of the lines coded according to \Vbl{species}. Describe the
differences between species.
%Heikert have high values on tars1,aede2, low on aede1,aede3.  
%Concinna has moderate values on tars1, high values on 
%aede2, aede1, aede3 %Heptapot has low values on
%tars1, aede2, moderate values on aede1, high aede3
\item Rearrange the order of the variables in the parallel coordinate
plot to better highlight the differences between species. 
  %tars1,aede2,aede1,aede3
\end{enumerate}

\item For the \Data{Italian Olive Oils}:
\begin{enumerate}
\item Generate a barchart of \Vbl{area} and a scatterplot of
\Vbl{eicosenoic} vs. \Vbl{linoleic}. Highlight \Vbl{South-Apulia}. How
would you describe the marginal distribution in \Vbl{linoleic} and
\Vbl{eicosenoic} acids for the oils from South Apulia relative to the other
areas?
\item Generate a grand tour of \Vbl{oleic}, \Vbl{linoleic},
\Vbl{arachidic}, and \Vbl{eicosenoic}. Describe the clustering in the 4D
space in relationship to difference between the three regions.
\end{enumerate}

\item For \Data{PRIM7}:
\begin{enumerate}
\item Sphere the data, that is, transform the data into principal
component coordinates.
\item Run a guided tour using the holes index on the transformed
data. Describe the most interesting view found.
% Should be the triangle with two arms pointing to each other.
\end{enumerate}
\item For \Data{Wages}, use linking by subject \Vbl{id} to identify a man:
\begin{enumerate}
\item whose wage is relatively high after 10 years of experience.
\item whose wages tended to go down over time.
\item who had a high salary at 6 years of workforce experience, 
but much lower salary at 10 years of experience. %2602
\item who had a high salary at the start of his workforce 
experience and a lower salary subsequently. %4736
\item with low wage at the start and a higher salary later. %6527
\item who has had a relatively high wage, and who has a graduate 
equivalency diploma. %9969
\end{enumerate}

\end{enumerate}

